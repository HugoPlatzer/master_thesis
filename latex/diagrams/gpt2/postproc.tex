\tikzstyle{circle math node}=[
%data node,
minimum size=5mm,
inner sep=0mm,
circle,
draw,
]

\tikzstyle{my caption node}=[
inner sep=0mm,
text width=5.5cm,
node distance=2mm,
align=right,
anchor=east,
font=\bfseries,
%draw,
]

\tikzstyle{large gap}=[
node distance=15mm and 0mm,
]

\colorlet{arrowcolor}{red!90!black}



\node[data node] (x) {$x_i$};
\node[block node, color 1, large gap, above=of x] (ln) {Final Layer Norm};
\draw[arrow style] (x) to (ln);

\matrix[matrix of math nodes, left delimiter={[}, right delimiter={]}, above=of ln] (xx) {
1.0 & -1.3 & \cdots & 0.2 \\
};

\draw[arrow style] (ln) to (xx);
);

\matrix[matrix of math nodes, node distance=6mm, left delimiter={[}, right delimiter={]}, right=of xx] (vtoken) {
-1.2 & 3.4 & \cdots & -0.1 \\
\cdots & \cdots & \cdots & \cdots \\
0.4 & -2.7 & \cdots & -1.0 \\
};

\begin{scope}[on background layer]
\node[fit node, fit=(vtoken-1-1)(vtoken-3-1), fill=red!20] (vt1) {};
\node[fit node, fit=(vtoken-1-2)(vtoken-3-2), fill=green!20] (vt2) {};
\node[fit node, fit=(vtoken-1-4)(vtoken-3-4), fill=blue!20] (vt3) {};

\node[fit node, fit=(xx-1-1), fill=red!20] (xx1) {};
\node[fit node, fit=(xx-1-2), fill=green!20] (xx2) {};
\node[fit node, fit=(xx-1-3)] (xxd) {};
\node[fit node, fit=(xx-1-4), fill=blue!20] (xx3) {};
\end{scope}

\node[circle math node, large gap, above=of xx1] (mul1) {$\cdot$};
\node[circle math node, large gap, above=of xx2] (mul2) {$\cdot$};
\node[circle math node, large gap, above=of xxd, draw=none] (muld) {$\cdots$};
\node[circle math node, large gap, above=of xx3] (mul3) {$\cdot$};

\draw[arrow style] (xx1) to (mul1);
\draw[arrow style] (xx2) to (mul2);
\draw[arrow style] (xx3) to (mul3);
\draw[arrow style, arrowcolor] (vt1.north) to (mul1.south);
\draw[arrow style, arrowcolor] (vt2.north) to (mul2.south);
\draw[arrow style, arrowcolor] (vt3.north) to (mul3.south);

\node[fit node, fit=(mul1)(mul3)] (mulf) {};
\node[circle math node, above=of mulf] (add) {$+$};

\draw[arrow style] (mul1) to (add);
\draw[arrow style] (mul2) to (add);
\draw[arrow style] (mul3) to (add);

\path let \p1=(ln.west), \p2=(ln.east), \n1={\x2-\x1} in 
node[block node, color 3, above=of add, minimum width=\n1] (sm) {Softmax};

\draw [arrow style] (add) to (sm);

\node[data node, large gap, above=of sm] (y) {$y_i$};

\draw [arrow style] (sm) to (y);

%\draw[arrow style] (xx1) to (vt1);
%\draw[arrow style] (xx2) to (vt2);
%\draw[arrow style] (xx3) to (vt3);

\matrix[matrix of math nodes, left delimiter={[}, right delimiter={]}, anchor=west] (ymat)
at (vtoken.west|-y)
{
0.03 & \textrm{"the"} \\
0.2 & \textrm{"is"} \\
\cdots & \\
0.01 & \textrm{"house"} \\
};



\coordinate (cap) at ($(x-|ln.west) + (-4mm, 0mm)$);

\node[my caption node] at (cap) {Output vector of final GPT2 block \\
$
\begin{aligned}
x_i \in \mathbb{R}^{n_{hidden}}
\end{aligned}
$};
\node[my caption node] at (cap|-ln) {One last layer normalization stage (same structure as in GPT2 blocks) };
\node[my caption node] at (cap|-sm) { Apply softmax to get a valid probability distribution };
\node[my caption node] at (cap|-mul1) { Blend columns of $v_{token}$ according to normalized $x_i$ (matrix-vector product) };
\node[my caption node, below=of vtoken, align=left, text width=4cm] { Token embedding matrix $v_{token}$ taken from the preprocessing stage. Each row represents the embedding vector for a token from the vocabulary. \\
$
\begin{aligned}
v_{token} \in \mathbb{R}^{n_{vocab} \times d_{hidden}}
\end{aligned}
$};
\node[my caption node] at (cap|-y) {Model-estimated probabilites for token at position $i+1$ \\
$
\begin{aligned}
y_i \in \mathbb{R}^{n_{vocab}}
\end{aligned}
$};