\section{Summary}

As a baseline, we trained a small transformer network on addition, multiplication and integer square root tasks, with different numbers of operand digits and training dataset sizes (\cref{baseline}).

We wanted to improve performance, in particular for the multiplication and integer square root tasks and thus attempted a number of changes to the model and dataset.

First, we explored whether changing the model size (number of layers, hidden vector sizes etc.) improved model performance (\cref{model_size_experiments}).
After that, we tried changing the sampling function for the training dataset to see whether prioritizing certain training samples improved performance (\cref{sampling methods}).
Then, we went on to change the actual training samples themselves (\cref{stepbystep}). For this, we tried two different strategies:
The first strategy was reversing the digits of the result for each training sample (\cref{resultreversal}).
The second strategy was augmenting each training sample with a scratchpad of intermediate steps of the arithmetic computation of that sample (\cref{scratchpad}).
Finally, we created heatmaps of model attention to get some insight into how the model processes input strings (\cref{heatmap}).

\subsection{Results}
When doing the baseline experiments (\cref{baseline}), we found out that addition was generally easily learned as long as the training dataset size was at least 10k (\cref{fig:baseline_add}), with accuracy of at least 99\% for 3, 5 and 10 digit addition when using a 100k training dataset. These results are in line with \cite{teaching}.

Multiplication proved much tougher, with only about 20\% accuracy achieved for 3-digit multiplication and zero accuracy for 5 and 10 digits (\cref{fig:baseline_mul}).

Integer square root, for a comparable operand size (single operand with twice as many digits as the two operands of addition or multiplication) proved to be somewhat in the middle in terms of difficulty. A 100k training dataset achieved 98\% accuracy for 6 digit integer square root and 64\% accuracy for 10 digit integer square root, with zero accuracy for 20 digit integer square root (\cref{fig:baseline_sqrt}).

Errors for addition and multiplication were generally single/multiple digits being off in the middle of the sum/product (\cref{error_analysis}). This matches the error patterns observed in \citepage{7}{lengthgen}, which shows digits in the middle of the result taking longer to converge and finishing at lower accuracy.
For square root, the errors where the result being off by one for numbers close to a square number.

Changing the model size did result in a model with bigger vectors or more layers needing fewer training steps to achieve the same accuracy (\cref{fig:model_size_add}, \cref{fig:model_size_mul}) but did not make a substantial difference in model capabilities, meaning it could still not learn multiplication with 5 or 10 digits, or integer square root with 20 digits.

Changing the digit sampling from uniform across the whole sample space to bit-uniform or digit-uniform provided little improvement either. This stands in contrast to \cite{positionmatters} where both uniform-digit sampling and result reversal provided some improvement in model accuracy.

Result reversal did improve accuracy for addition when using a 100k training dataset: from 98-99\% to 100\% (\cref{tbl:reverse_add}). While it did increase reliability for addition, it did not help with making large multiplication or integer square root learnable.

Scratchpads led to big improvements for addition and multiplication: Addition could be learned perfectly with a 10k and 100k training dataset, and even for a 1k training dataset, accuracies ranged between 85\% and 99\% (\cref{tbl:scratchpad_add}). For multiplication, accuracies of 94\% for 3-digit multiplication, 63\% for 5-digit multiplication, and 22\% for 10-digit multiplication could be achieved. Interestingly, these numbers were obtained with a 10k training dataset, with slightly lower numbers achieved for the 100k training dataset. For integer square root, scratchpad performed worse than baseline (\cref{tbl:scratchpad_sqrt}) - this could be due to our square root scratchpad not containing each and every intermediate step due to size constraints.

When looking at the heatmaps for model attention, we could see some meaningful relationships between input and output digits in the first layer, especially for addition, corresponding to how addition would be done by hand (\cref{fig:heatmap_add_full}, \cref{fig:heatmap_add_simple1}). What happened in later layers was not obvious in the visualization. When using result reversal and scratchpad, there was a clear connection between input digits and these same digits in the intermediate steps of the computation, but no clear pattern of the rest of the computation (\cref{fig:heatmap_add_rev}, \cref{fig:heatmap_add_scratch}).

In summary, the best results we could achieve for the arithmetic functions we studied were 100\% accuracy for 3, 5 and 10 digit addition, 94\% for 3-digit multiplication, 63\% for 5-digit multiplication, 22\% for 10-digit multiplication, 98\% for 6-digit integer square root, 64\% for 10-digit integer square root and 0\% for 20-digit integer square root.
For addition and multiplication, these results were obtained when using scratchpads, for integer square root, they are from the baseline model.

On the one hand, these results could likely be made even better by longer training when using scratchpads (we constrained epochs to keep training time manageable with the long scratchpad strings).
On the other hand, using scratchpads also requires knowledge of meaningful intermediate steps - how to compute the output from the input step by step. A key advantage of machine learning over traditional algorithms is that this knowledge of how to compute the result is generally not required: The model is provided input/output pairs and figures out the rest. This advantage is lost when needing to build a detailed scratchpad for the training samples.



\subsection{Future work}

Some future areas of research that might be worth exploring are:

It is unclear whether the stopping condition we used does lead to premature termination of training - looking at accuracy curves suggests this might be a possibility. Focusing not only on loss but also accuracy might ensure training does not stop before the model reaches maximum accuracy.

In the same way, training when using scratchpad in our case was limited by resources available to us, training for more epochs would allow to see the actual limitations of scratchpad training for multiplication on our model better.

We did not analyze variance between multiple training runs as even a single training run took substantial resources. It would however be interesting to see if a task that normally fails to converge occasionally does converge or vice versa.

It would be interesting to see whether a second model could learn to predict where the main model makes a mistake and try to correct it, especially for longer scratchpads

It would be worth combining result reversal with uniform-digit sampling to try and replicate the good  results for multiplication in \cite{positionmatters}.

The good results of basic feed-forward networks in \cite{visual} would make it worthwhile to explore their limits, especially with regards to longer scratchpad sequences. It would be worth finding out at which point transformers start having an advantage over older, more basic networks. Other structures like 1-dimensional CNNs or recurrent networks could be compared too.

Using a simpler, more straightforward scratchpad for multiplication as in \cite{implicit} would reduce sequence length and thus accelerate training. Performance between such a bare-bones scratchpad and the more elaborate one we are using should be compared.

At the same time, a better scratchpad for integer square root should be devised. It must be reasonably short, yet not leave big chunks like the computation of squares without intermediate steps.

The inner mechanisms of the model could be explored further - our heatmaps show some clear, logical relationships between input and output in the first layer already - but it does not explain the whole "algorithm" learned by the model. Analysis would have to focus on the more opaque patterns in later layers of the model.