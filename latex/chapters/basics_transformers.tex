\section{Background}

\subsection{Transformer models}

There are a variety of different albeit similar structures that have been used since the "Attention is all you need" paper to build effective transformer networks. In particular, decoder, encoder-decoder, and encoder models need to be differentiated.

Encoder models such as BERT \cite{devlin2019bert} transform any given text into a vector of fixed size. This vector in the ideal case captures the essence of the text's meaning and can thus be used for text classification, sentiment analysis etc.

Encoder-decoder models like the original transformer architecture as described in \cite{allyouneed} use the vector generated by the encoder to start generating new text. This can be used for example by translation models or for answering questions.

Decoder models like the GPT models \cite{radford2018improving} generate new text to extend some given starting text. This can be used for example to write a story when given a first paragraph. But with the right training, they can be used for the same tasks as encoder and encoder-decoder models as can be seen in the next section.

\subsubsection{Transformer decoder models}

The purpose of a decoder transformer model is to have it complete a sequence of tokens. These tokens may represent a letter of some alphabet, fragments of words, entire words, keywords of a programming language, even decimal digits, byte values or the bits 0/1...
They could even represent color values, grid coordinates etc. Natural language modeling however is the most common scenario.

The model can be trained on token sequences generated from natural language taken from books, news articles, online communication platforms etc. This would make the model useful for writing stories, filling gaps in text etc.
Or it could be trained on pieces of source code, which would allow the model to act as a code completion assistant.
The model would be given a piece of text that starts a story the user wants to have completed, or a piece of code that only has a function defined with comments yet. A well-trained model would then completed the code or story in a useful way so the rest of the story matches the theme set in the starting paragraph, or the code generated does what the comments and function signature above says it does.

The training texts could also all be in a structured question-answer pair form, allowing the model to answer questions. A few examples:
\label{transformer:examples}

\begin{itemize}
\item Translation tasks: "en: The house is big.  de: Das Haus ist groß."
\item Classification tasks: "text:  what a cool product, would buy again  sentiment: happy"
\item Mathematical tasks: "21*2=42" or "123: odd number" / "124: even number" \citepage{3}{nocalculator}
\end{itemize}

When the models is well-trained on these question-answer pair texts, it can be asked a question by simply leaving out the answer part at the end of the text, like "21*3=". \cite{alammar-transformer} A good model would be able to generate the correct answer for similar questions by inference from the many examples it has seen, even when the exact question posed is not in its training data.

\subsubsection{Beyond text}

Apart from text-based tasks, transformer models are also capable in domains like image classification:
A 2D image can be flattened into a sequence of tokens / vectors which can then be processed in a normal transformer model. The resulting image classifier achieves similar results to state-of-the-art CNN-based classifiers as can be seen in \citepage{3,5}{worth16words}.

Transformers can even be successful at generating images from text prompts: The approach presented in \cite{cogview} involves using a regular tokenizer for the text prompt and a separate discrete autoencoder to transform the image into tokens.

Board games such as chess and go can also easily be tackled by transformers, as the board state and moves made can be transformed into text or represented by specialized tokenization schemes.
In \cite{grandmasterlevelchess}, a transformer decoder model is trained to generate best moves or score board positions based on a large dataset of human chess games. The resulting chess playing engine manages grandmaster-level performance in blitz chess against human opponents \citepage{5}{grandmasterlevelchess}.

There have been minor modifications to the model architecture, data representation and training regime among the diverse tasks outlined above. Also the performance of the resulting models is generally good but not quite state-of-the art. Nonetheless, having one model architecture that can achieve strong performance across a diverse set of domains seems promising as a step towards artificial general intelligence.


\subsubsection{Tokenization}

Most transformer models preprocess input text using a so-called tokenizer, which is a model that was trained to group text into meaningful chunks out of a limited vocabulary called tokens to provide for an efficient representation of language \cite{subwordunits}. The tokenizer transforms the text into a sequence of integers, which is then processed by the transformer model, it outputs token IDs which are then converted back into text.

Compared to just using raw bytes as model inputs, a good tokenizer allows for shorter sequence lengths (as a token typically represents multiple characters/bytes), meaning faster generation and greater maximum length \cite{tokenizerchoice}. Nonetheless, there are capable models such as Google ByT5 \cite{xue2022byt5} that do not need a tokenizer as they operate directly on byte values.
Compared to manually specifying a vocabulary (for example contents of a dictionary) tokenization is more flexible, having no issues with unknown words or large composite words (any possible byte sequence can still be represented) and being adaptable to the text corpus the model should be used on.

To train a tokenizer, the byte-pair-encoding (BPE) algorithm \cite{subwordunits} is often used, which also has many applications in data compression. It works by finding the most common byte pair in the data, then creating a new symbol that substitutes for these two bytes. The process is then repeated, considering pairs of symbols as a single "new byte value", until every byte/symbol pair occurs only once, or in the case of transformer models until a compromise between token vocabulary size and characters per token is reached.

Apart from the regular token vocabulary, most tokenizers also contain some special token values that will not be found in the input text. They are used for example to delimit individual pieces of text when they are fed to the model, or separate parts of a questions/answers pair \cite{HuggingFaceTokenizers}.


\subsubsection{Input}

More formally, the input to the model consists of a sequence of integers (token IDs) $x_1 \ldots x_l$, where $l$ is the length of the token sequence.
Generally, transformer models place a limit on the length of the sequence due to their internal structure, however models with no such limitation have also been proposed. \cite{su2023roformer}

Each integer $x_i$ references one of the possible tokens of the model $(1 \leq x_i \leq n_{vocab})$. $n_{vocab}$ is the size of the token vocabulary, which depends on the tokenization process used for the model \citepage{2}{allyouneed}.

\subsubsection{Output}

\label{transformer:output}

For each position in the input sequence, the model outputs a probability distribution, predicting the next token based on the input tokens up to this position. 

The output of the model is a sequence of probability distributions $y_i$ $(1 \leq i \leq n_{pos})$. $y_i \in \mathbb{R}^{n_{vocab}}$, $0 \leq {y_i}_j \leq 1$ \citepage{5}{allyouneed} \cite{HuggingFaceGPT2}.
$y_i$ should tell which token most likely follows after all the tokens $x_1 \ldots x_i$.

\subsubsection{Training}

During training, all output positions are useful as they allow to optimize the model to correctly predict $x_2$ based on $x_1$,
$x_3$ based on $x_1$ and $x_2$, $x_{n_{pos}}$ based on $x_1 \ldots x_{n_{pos}-1}$, all in a single training step. This greatly reduces training time for longer sequences \cite{alammar-gpt2}.

\subsubsection{Generation}

When using the trained model to generate text, only the last output distribution, $y_{n_{pos}}$, is of interest, because
it tells the next token after the given input sequence. \cite{Mao2021Autoregressive}

This is used to generate one new token at a time (autoregressive decoding). The new token is picked from the distribution either by choosing the most likely token (greedy decoding), random sampling according to the probability distribution, or more sophisticated methods such as top-p and top-k sampling \cite{ippolito2019comparison}.

The newly generated token $x_{n_{pos}+1}$ is appended to the input sequence and the process is repeated. This process is repeated until a special end-of-sequence token is generated, a specified maximum length is reached, or some other stop condition is triggered \cite[sect. "Generation"]{HuggingFaceGPT2}.

\subsubsection{User interaction}

In 2023, OpenAI released their most successful product yet, ChatGPT \cite{openai_chatgpt_2022}. This online service allows users to chat with an artificial intelligence assistant which can help with reciting facts, writing texts and code, giving advice of all sorts etc.

The principle by which transformer decoder models can act as chat assistants is simple: The model is trained on conversations like this:

\begin{verbatim}
    User: What color is the sky?
    Assistant: The sky is blue.
\end{verbatim}

The model can then become the helpful assistant by filling in the assistant's messages:

\begin{verbatim}
    User: What color is a banana?
    Assistant:
\end{verbatim}

and the model would generate a response like "A banana is yellow". For more details on how generic language models can be tuned to follow instructions or conversation, see \cite{yi2019coherent} and \cite{ouyang2022training}.

\subsubsection{Generative pre-training}

The GPT2/GPT3/ChatGPT transformer models have demonstrated the effectiveness of a technique used to train language models called generative pre-training (GPT). This technique was first popularized in \cite{improvinglu} and consists of two phases:

In the pre-training stage the model is first trained on a large corpus of text in multiple languages, including websites, books, articles, chats, source code etc. In this manner, a huge amount of training data can be collected without the need for careful manual selection / preparation of question-answer pairs. The training data is split into appropriate chunks and the model learns to predict (generate) the next word based on the previous ones for each chunk as a form of unsupervised learning \citepage{3}{improvinglu}.

After sufficient pre-training, the model is then fine-tuned on question-answer pairs of the specific task the model should be able to solve, like the examples presented in \ref{transformer:examples} or even human-assistant chats on a wide range of topics.

Due to the well-known phenomenon of transfer learning \cite{transferlearning}, the pre-trained model will have already internalized the structure of natural language, source code, maths etc. 
This allows a relatively small amount of hand picked fine-tuning examples to make it significantly stronger compared to doing the same on a freshly initalized model \citepage{8}{improvinglu}.

\subsection{Transformer architecture by example: OpenAI GPT2}

A good example of a successful decoder model is OpenAI's GPT2 \cite{unsupervisedmultitask} \cite{OpenAI2019BetterLM} \cite[source code]{HuggingFaceGPT2}: It was released in 2019 and demonstrated the ability to write realistic articles based on a given starting prompt. This model is useful for explaining the details of the transformer architecture, because its weights and code were released, its proven effectiveness in generating meaningful text, and due to the fact that its even more powerful successors have not experienced significant architectural changes \cite{OpenGenus2023GPTComparison}.

\subsubsection{Architecture}

\includediagram{diagrams/gpt2/overview}{Overview of the GPT2 model (gpt2-small, 12 blocks, $n_{hidden} = 768$).}

For understanding GPT2, it is important to realize the multiple dimensions at play inside the model \cite{alammar-gpt2} \citepage{6}{allyouneed}:
There are multiple positions, as many as there are tokens in the input sequence: $n_{pos}$.
Each position is a vector of fixed dimension having $n_{hidden}$ real numbers as components.
Also, in practice, multiple input sequences can be processed in parallel for better efficiency (batching).
For simplicity, in this explanation we only consider a batch of size 1, i.e. a single input sequence.

In practice, when libraries implement this kind of model, the steps described now are done using matrix operations
for optimal parallelism, or optimized in other ways. The step-by-step manner used to describe the operations,
especially the attention mechanism, is helpful because it is easiest to understand, but real-world code
utilizing e.g. GPU parallelism would use better optimized algorithms to achieve the same output \cite{choi2023unleashing} \citepage{524}{yu2022orca}. 

For each computation step described in the following sections, $x_i$ describes the input vector to that step (at position $i$), while $y_i$ describes the output vector
at that step. In the previous section, $x_i$ was used to refer to the input (token ID sequence) and $y_i$ to the final output of the model (probabilities).
The same symbols $x_i$ and $y_i$, however, will also be used to describe the inputs and outputs of each of the small steps/parts of the model.
They should be seen as fresh independent symbols for each computation step described. In general, the $y_i$ (output) of one step is the following step's $x_i$ (input).
This also applies to other symbols like those describing weight matrices, etc. These symbols are independent across different processing steps.

The letter $i$ in the explanation always refers to the position within the sequence. In general, the same function is computed with the same learnable parameters
for the inputs at each position independently. Only for the attention step, the inputs of previous positions are also contributing to the output \cite{alammar-gpt2}.

\subsubsection{Tokenization}

\includediagram{diagrams/gpt2/tokenizer}{Operation of the GPT2 tokenizer on an example sentence.}

\label{gpt2:tokenizer}

GPT2 uses a BPE tokenizer with a vocabulary of 50257 tokens \citepage{4}{unsupervisedmultitask} \cite{HuggingFaceGPT2}.

\subsubsection{Number of parameters}

\label{gpt2:parameters}

To make the model bigger (more memory and compute intensive, but more accurate generation) or smaller (faster, but lower quality output), two hyperparameters can be adjusted \citepage{4}{unsupervisedmultitask} \citepage{5}{radford2018improving}: The vector size for each position can be changed just like the number of transformer blocks. The number of attention heads is usually adapted so the size of each head remains around 64 \citepage{4}{unsupervisedmultitask} \cite{hfpretrained}.
OpenAI GPT2 was released in four versions with different parameter counts:

\includeTable
{
    name; n_params; n_blocks; d_hidden; n_heads; d_head\\
    gpt2-small; 117M; 12; 768; 12; 64 \\
    gpt2-medium; 345M; 24; 1024; 16; 64 \\
    gpt2-large; 774M; 36; 1280; 20; 64 \\
    gpt2-xl; 1558M; 48; 1600; 25; 64 \\
}
{tab:gpt2-sizesX}
{
    Versions of GPT2 released with different parameter counts \cite{hfpretrained}: $n_{blocks}$ is the number of GPT2 blocks, $d_{hidden}$ is the vector size at each position, $n_{heads}$ is the number of attention heads and $d_{head}$ is the size of each attention head.
}
{%
    columns/name/.style={column name={Model size}},
    columns/n_params/.style={column name={\#Parameters}},
    columns/n_blocks/.style={column name={$n_{blocks}$}},
    columns/d_hidden/.style={column name={$d_{hidden}$}},
    columns/n_heads/.style={column name={$n_{heads}$}},
    columns/d_head/.style={column name={$d_{head}$}}
}

\subsubsection{Processing steps}

When using the model to extend a given starting text (prompt), the following steps are taken \cite{alammar-gpt2} \citepage{113}{choi2023unleashing}:

\begin{itemize}
\item The tokenizer transforms the text into a sequence of token IDs (integers).
\item The preprocessing stage generates an initial vector for each position in the token sequence
\item This sequence of vectors is passed through the blocks of the model, each block outputting a new vector sequence of equal dimensions
\item After the last block, the postprocessing stage generates a token probability distribution for each position
\item Based on the probabilities at the last position, a token ID is picked. This new token is appended to the input token sequence
\item The previous steps are repeated until a stopping condition is reached
\end{itemize}

The token ID sequence is then converted back into text using the tokenizer vocabulary. This results in the original text plus new text appended by the model. If the model is well trained to the language/task, the generated text will match the given starting text (prompt) both syntactically and semantically.

\subsubsection{GPT2 preprocessing}
\label{gpt2_preproc}

\includediagram{diagrams/gpt2/preproc}{GPT2 preprocessing stage on an example token sequence. Based on the token position and value, embedding vectors are summed to give the starting vector at this position.}

Transformer networks do not operate with integer token IDs, but with state vectors from $\mathbb{R}^{d_{hidden}}$. $d_{hidden}$, the dimension of the state vectors at each position, is a parameter fixed at the time of model construction (\ref{gpt2:parameters}).

This means the token sequence needs to be converted into a sequence of such vectors before feeding it through the model. This happens during the preprocessing stage.
For this, GPT2 has a learnable token embedding matrix $v_{token}$.  Because the GPT2 tokenizer has a vocabulary size of 50257 (\ref{gpt2:tokenizer}), this matrix is of size $50257 \times d_{hidden}$.

\begin{samepage}

Furthermore, because the processing blocks of a transformer model have no positional awareness, information about the position of a vector within the sequence needs to be included as well \citepage{6}{allyouneed}.
In GPT2, this is done by another learnable embedding matrix $v_{pos}$ storing a vector for each position up to some limit, which will become the maximum sequence length the model can handle \citepage{5}{improvinglu}  . For GPT2, this maximum sequence length has been fixed at 1024 \citepage{4}{unsupervisedmultitask} \cite{HuggingFaceGPT2}.
The positional embedding matrix $v_{token}$ thus has a size of $1024 \times d_{hidden}$.
The starting vector $y_i$ for a token value $x_i$ at position $i$ is then given as \citepage{3}{allyouneed}:

$$y_i = v_{token}(x_i) + v_{pos}(i)$$

\end{samepage}


\subsubsection{GPT2 block}

\includediagram{diagrams/gpt2/block}{Overview of the GPT2 block components. This example shows 3 positions $x_1 \ldots x_3$, but the actual model is flexible regarding sequence length.}

Each block has an attention sub-block, an MLP(multi-layer perceptron) sub-block as well as layer normalization stages and residual connections \citepage{3}{allyouneed} \citepage{4}{improvinglu}.
The operations of the block are executed for each position $i$, but the learnable parameters are the same independent of the position. Different blocks however have different parameters.

The learnable parameters of each block are \cite{alammar-gpt2}:

\begin{itemize}
\item weight and bias of the linear transformation that generates query, key, value vectors from the input vector
   at the start of the attention mechanism: $W_q, b_q, W_k, b_k, W_v, b_v$
 \item weight and bias of the linear transformation "projection" at the end of the attention mechanism: $W_p$, $b_p$
 \item weight and bias of the two linear transformations within the MLP sub-block
 \item weight and bias for each of the two layer normalizations (at the end of each layer norm there is a linear transform): $W_{ln_1}, b_{ln_1} W_{ln_2}, b_{ln_2}$
\end{itemize}

\subsubsection{GPT2 block: Layer normalization}

\includediagram{diagrams/gpt2/layernorm}{Detailed view of the layer normalization stage of the GPT2 block.}

$$x_i = x_{i_j} (1 \leq j \leq n_{hidden})$$

$$x'_{i_j} = \frac {x_{i_j} - \mu(x_i)} {\sqrt {\sigma^2(x_i)+\epsilon}} $$

$$y_{i_j} = a_j x'_{i_j} + b_j$$

$\mu$ and $\sigma^2$ are the mean and variance of a vector, $\epsilon$ is a small positive real constant added for numerical stability, and $a$ and $b$ are learnable gain and bias vectors of dimension $n_{hidden}$.

Layer normalization subtracts the mean and divides by the standard deviation among all features of the input vector.
After this normalization, it applies a learned gain and bias value for each component of the vector.
This is in contrast to batch normalization, where each individual feature $x_{i_j}$ is normalized along the batch dimension \citepage{1-2}{ba2016layer}.
Layer normalization has the advantage of not suffering from statistical noise when the batch size is too small.

The normalization happens independently for each vector $x_i$ (i.e. at each position $i$), but the learnable parameters $a$ and $b$ (like all learnable parameters of the model) are the same for each position.

\subsubsection{GPT2 block: Residual connection}

\begin{samepage}

$$y_i = x_i + f(x_i)$$

as opposed to $y_i = f(x_i)$ without a residual connection \cite[p. 3]{allyouneed} \cite{residual}.

A residual connection allows the input to "bypass" the learned function of the model, by adding the input to the output vector.
Such connections are found in almost all models having many steps/layers, because they were found to greatly help when training deep models.
This is because when using residual connections, the input data does not need to flow through all computation steps before reaching a given point, instead it can bypass some of them, allowing later computations to access earlier data \citepage{1-2}{residual}.

\end{samepage}

\subsubsection{GPT2 block: Attention mechanism}
\label{gpt2_attn}

\includediagram{diagrams/gpt2/attention_multihead}{Structure of the Multi-head attention mechanism. This example uses two positions $x_1, x_2$ and two attention heads $h_1, h_2$. The actual model can in theory handle arbitrary sequence lengths and also has more attention heads.}



At its core, GPT2, like previous transformer models \citepage{3}{allyouneed}, uses a mechanism called Multi-Head Attention \citepage{4}{allyouneed}.

The general idea of an attention mechanism is to compare the vector at one position $i$ with those at previous positions by generating query/key/value vectors at each position, matching the query at position $i$ with the keys of previous positions, then blending the value vectors accordingly to generate the output at position $i$ \cite{alammar-transformer} \cite{alammar-gpt2}.

This is the only part of the entire model where the output at a given position also depends on other (previous) positions. All other components of the model process each position independently.
Nonetheless, the learnable parameters of the attention mechanism, which are linear transformations to generate query, key, value vectors and project the output, are independent of position.

\includediagram{diagrams/gpt2/attention_dot}{Structure of the Scaled Dot-product attention mechanism. This diagram illustrates the computation of the output position $a_3$. The other output positions $a_i$ are computed in a similar fashion, replacing $q_3$ in the diagram with $q_i$ and considering $k_1 \ldots k_i$ and $v_1 \ldots v_i$, giving the diagram $i$ columns.}

For each position $i$ $(1 \leq i \leq n_{pos})$ the following steps are performed \citepage{4}{allyouneed} \cite{alammar-transformer} \cite{alammar-gpt2}:


\begin{itemize}



\item Three vectors called query, key and value are computed using a linear transformation with bias. Its parameters are learned during model training.
 
 $$q_i = W_q x_i + b_q$$
 $$k_i = W_k x_i + b_k$$
 $$v_i = W_v x_i + b_v$$

\item The query, key and value vectors $q_i, k_i, v_i$ are split into $n_{heads}$ segments of equal size $d_{head} = \frac{d_{hidden}}{n_{heads}}$. $q_{i_s}, k_{i_s}, v_{i_s} \in \mathbb{R}^{d_{head}}$. The following steps are performed for each segment $s$ $(1 \leq s \leq n_{heads})$. This is called multi-head attention and was found to be beneficial to model performance comparing to just using a single head \citepage{4-5}{allyouneed} \cite{alammar-gpt2}.
Multi-head attention can also be described as running the Scaled Dot-product attention mechanism $n_{heads}$ times in parallel with each query/key/value vector having the size $d_{head} = \frac{d_{hidden}}{n_{heads}}$ with different learnable parameters for each head, then concatenating and projecting the outputs for all heads at each position \citepage{4-5}{allyouneed}.



\begin{samepage}
\begin{itemize}

\item The query vector $q_{i_s}$ is compared with the key vector of each position $j$  $(1 \leq j \leq i)$ using dot product to generate a score value $a_{ij_s}\in \mathbb{R}$.

$$a_{ij_s} = \frac {1} {\sqrt{d_{head}}} \ dot(q_{i_s}, k_{j_s})$$

The scaling factor $\frac {1} {\sqrt{d_{head}}}$ helps improve gradient stability \citepage{4}{allyouneed}.

Positions after $i$ $(j > i)$ are not considered because of the autoregressive nature of the model, i.e. the model is going to be used to predict the token at position $i+1$ based on the tokens up to position $i$. While the entire sequence is available during training, allowing for an efficient training process, during evaluation only the tokens up to the current position $i$ will be available so this is what should be learned \citepage{5}{allyouneed}.


\item The scores $a_{ij_s}$ are normalized using softmax so their sum is 1, this is required for the next step.
   $$ softmax(a_{ij_s}) = \frac {e^{a_{ij_s}}} {\sum_{1 \leq j' \leq i} {e^{a_{ij'_s}}}} $$


\item Using the scores from before as weights, the output vector $o_{i_s}$ is "blended together" from the value vectors $v_{j_s}$ $(1 \leq j \leq i)$ \cite{alammar-gpt2}:
    $$o_{i_s} = \sum _{1 \leq j \leq i} softmax(a_{ij_s}) \ v_{j_s}$$

\item This so-called "Scaled Dot-product attention mechanism" is further illustrated in \cref{diagrams/gpt2/attention_dot}.

\end{itemize}
\end{samepage}


\item The output vectors of the segments are concatenated:
   $$o_i = concat _{1 \leq s \leq n_{heads}} \ o_{i_s}$$



\item Finally, a linear transformation with bias (the original transformer paper mentions such a transform, but with no bias \citepage{5}{allyouneed}) is applied on the concatenated vector \cite{alammar-gpt2}. Its parameters are learnable.
   $$y_i = W_p o_i + b_p$$

\end{itemize}

\subsubsection{GPT2 block: Multi-layer perceptron}

\includediagram{diagrams/gpt2/mlp}{Multi-layer perceptron stage of the GPT2 block. It is a simple feed-forward neural network with two linear steps with bias and an activation function in between.}

This is a simple feed-forward neural network with one hidden layer \cite[p. 5]{allyouneed}.
As opposed to the original transformer, GPT2 uses the Gaussian Error Linear Unit (GELU) activation function \citepage{5}{improvinglu}, which is defined as follows \cite{gelu}: $gelu(x) = x \ P(X<x)$ when $X$ follows a normal distrbution with mean 0 and variance 1.

$$x'_i = gelu(W_c x_i + b_c)$$
$$y_i = W_p x'_i + b_p$$

\subsubsection{GPT2 postprocessing}

\includediagram{diagrams/gpt2/postproc}{Detailed example of the GPT2 postprocessing stage. $x_i$, the output of the last GPT2 block, is converted into token probabilities.}

\label{gpt2:postproc}

GPT2 uses a "language modeling head" \cite{HuggingFaceGPT2} to convert the outputs of the last block into token probabilities. This works in a similar way as the unsupervised learning objective described in \citepage{3}{improvinglu}. The same token embedding matrix $v_{token}$ that is also used in the preprocessing stage to convert tokens into vectors is used here (using matrix-vector product) to transform a $d$-dimensional vector into a $n_{tokens}$ dimensional vector, softmax then gives a probability distribution \cite{github-hf} \citepage{3}{improvinglu}.

Before this conversion, a final layer normalization step is performed \citepage{4}{unsupervisedmultitask}.


\subsubsection{Text generation}

\includediagram{diagrams/gpt2/generation}{Text generation using the GPT2 model for the example starting text "where is". The rightmost probability distribution is used to pick the next token, after which the data, augmented with this new token, is fed through the model again.}

During evaluation time, the goal is not just generate one token, but to meaningfully complete the input sequence, possibly using many more tokens.
Nonetheless, the new tokens are generated one at a time. When the input is $x_1 \ldots x_n$, the last position of the output, $y_n$, is used to generate a new token $x_{n+1}$.
Because $y_n$ is a probability distribution, an integer value (the ID of the token $x_{n+1}$) must be sampled from it. This can be done using uniform sampling,
top-k, top-p, or trivially by picking the highest probability, which would mean the model always generates the same output from a given input \cite{hf-howtogeneratetext}.
This could be fine for e.g. a mathematical task, but for more creative tasks like story generation, diversity of generated candidates is useful.

The sampling of new tokens happens until a stop token is generated, the model maximum $n_{pos_{max}}$ is reached or some other abort condition is met. \cite[p.2 ]{allyouneed}


\subsubsection{Training}

GPT2 uses an unsupervised training objective similar to \citepage{3}{improvinglu} where a large text corpus is used to make the model predict the next token based on the tokens up to this point.

Cross-entropy loss is used on each output position $y_i$ to measure the deviation between the output $y_i$ and the ideal $y'_i$,
where $y'_i$ represents the probability distribution of a guaranteed token with ID $x_{i+1}$ \citepage{2}{unsupervisedmultitask} \cite{HuggingFaceGPT2}.
As explained before (\ref{transformer:output}), the model should predict the next input token at each output position. This means, when having a training sample sequence $x_1 \ldots x_n$, the inputs to the model would be $x_1 \ldots x_{n-1}$, whereas the ideal outputs would be $x_2 \cdots x_n$ (the inputs shifted by one position) \citepage{3}{radford2018improving} \citepage{7}{reasonoverscenegraphs}.

Standard initialization techniques (for example uniform initialization) can be used for the weight and bias matrices \citepage{5}{improvinglu}.

GPT2 uses the Adam optimizer \citepage{5}{improvinglu}.
A naive stochastic gradient optimizer will not achieve good results when training transformer networks like GPT2, thus more advanced optimizers like Adam need to be used to make the model converge to a degree where it becomes useful \cite{adambeatssgd}.