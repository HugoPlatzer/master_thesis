\documentclass{article}

%%% packages
\usepackage{lmodern}
%\usepackage{indentfirst}
\usepackage{amsmath}
\usepackage{amssymb}
\usepackage{graphicx}
\usepackage{svg}
\usepackage{amsmath}
\usepackage{amssymb}
\usepackage[labelfont=bf]{caption}
\usepackage{ifthen}
\usepackage[a4paper]{geometry}
\usepackage{booktabs}
\usepackage{pgfplotstable}
\usepackage{colortbl}
\usepackage{float}
\usepackage{makecell}
\usepackage{hyperref}
\usepackage{xparse}
\usepackage{cleveref}
\usepackage{tikz}
\usepackage{listings}
\usepackage[
    style=numeric,
    backend=biber,
    date=long,
    defernumbers=true,
    sorting=none,
    urldate=long]
    {biblatex}
%\usepackage{letltxmacro}\LetLtxMacro{\cite}{\parencite}

\addbibresource{biblio.bib}
\AtEveryBibitem{%
  \clearfield{eprint}%
  \clearfield{doi}%
}


%%% package configuration
\hypersetup{
    colorlinks=true,
    linkcolor=blue,
    urlcolor=blue
}

\newcommand{\highlightbox}[2]{%
  \begingroup%
  \setlength{\fboxsep}{0pt}%
  \colorbox{#1}{\strut\ttfamily#2}%
  \endgroup%
}

\lstset{
    keywords={while, for, if, else, return},
    keywordstyle=\bfseries,
    basicstyle=\ttfamily,
    moredelim=[is][\highlightbox{red!20}]{!}{!},
    columns=fullflexible,
    keepspaces
}

\usetikzlibrary{positioning}
\usetikzlibrary{fit}
\usetikzlibrary{backgrounds}
\usetikzlibrary{shapes}
\usetikzlibrary{matrix}
\usetikzlibrary{calc}
\usetikzlibrary{math}
\usetikzlibrary{arrows.meta}
\tikzstyle{line style}=[
line width=0.3mm,
]

\tikzstyle{arrow style}=[
->,
line width=0.3mm,
shorten <= 1.5mm,
shorten >= 1.5mm,
]

\tikzstyle{short arrow style}=[
->,
line width=0.3mm,
shorten <= 4mm,
shorten >= 4mm,
]

\tikzstyle{connect arrow style}=[
->,
line width=0.3mm,
]

\tikzstyle{behind arrow style}=[
->,
%dashed,
rounded corners,
line width=0.3mm,
shorten <= 1mm,
shorten >= 1mm,
]


\tikzstyle{block node}=[
rounded corners,
font=\bfseries,
node distance=6mm and 0mm,
text width=3cm,
align=center,
]

\tikzstyle{data node}=[
inner sep=0mm,
node distance=6mm and 0mm,
align=center,
%draw,
%text height=3mm,
%text depth=2mm,
]

\tikzstyle{sum node}=[
circle,
draw,
minimum size=7mm,
node distance=6mm and 0mm,
append after command={
node at (\tikzlastnode) {$\textbf{+}$}
},
]


\tikzstyle{caption node}=[
font=\bfseries,
node distance=0mm and 3mm,
%text width=5cm,
%text height=6mm,
%align=right,
%draw,
]

\tikzstyle{small caption node}=[
font=\footnotesize,
]

\tikzstyle{highlight node}=[
ellipse,
minimum height=8mm,
minimum width=12mm,
fill=gray!30,
draw,
line style,
]

\tikzstyle{insight node}=[
fill=gray!30,
draw,
line style,
]

\tikzstyle{fit node}=[
inner sep=0mm,
%draw,
%draw=red,
]

\tikzstyle{matrix style}=[
node distance=7mm and 7mm,
%draw=red,
]


\tikzstyle{color 1}=[fill=blue!30]
\tikzstyle{color 1a}=[fill=blue!30]
\tikzstyle{color 1b}=[fill={rgb:blue,1;green,1;white,3}]
\tikzstyle{color 2}=[fill=red!30]
\tikzstyle{color 3}=[fill=orange!30]

\pgfplotsset{compat=1.18}

\definecolor{mutedred}{rgb}{1.0, 0.3, 0.3}
\definecolor{mutedgreen}{rgb}{0.3, 1.0, 0.3}

%%% custom commands
\newcommand{\citepage}[2]{\cite[p.~#1]{#2}}

\newcommand{\includediagram}[2]{
    \begin{figure}[H]
    \centering
    
    \begin{tikzpicture}[framed]
    \input{#1}
    \end{tikzpicture}
    
    \caption{#2}
    \label{#1}
    \end{figure}
}

\newcommand{\mapvaluetocolor}[1]{%
    \pgfmathsetmacro{\colorvalue}{(1-(#1))*100}%
    \edef\temp{\noexpand\cellcolor{mutedred!\colorvalue!mutedgreen}}%
    \temp
}

% arguments: table content, label, caption, additional styling options
\newcommand{\includeTable}[4]{
    \begin{table}[H]
        \centering
        \pgfplotstabletypeset[
            string type,
            row sep=crcr,
            col sep=semicolon,
            assign column name/.style={/pgfplots/table/column name={\textbf{##1}}},
            every head row/.style={
                before row=\toprule,
                after row=\midrule,
            },
            every last row/.style={
                after row=\bottomrule
            },
            every column/.style={
                column type={r}
            },%
            #4,
        ]{#1}
        \caption{#3}
        \label{#2}
    \end{table}
}

% arguments: csv file, additional styling options
\newcommand{\includeAccuracyTableCore}[2]{
    \pgfplotstabletypeset[
        string type,
        col sep=comma,
        assign column name/.style={/pgfplots/table/column name={\textbf{##1}}},
        every head row/.style={
            before row=\toprule,
            after row=\midrule,
        },
        every last row/.style={
            after row=\bottomrule
        },
        every column/.style={
            column type={r},
            postproc cell content/.code={%
                \pgfkeyssetvalue{/pgfplots/table/@cell content}{\mapvaluetocolor{####1} ####1}%
            },
        },
        columns/name/.style={
            column name={},
            postproc cell content/.code={}
        },
        #2
    ]{#1}
}

% arguments: csv file, label, caption, additional styling options
\newcommand{\includeAccuracyTable}[4]{
    \begin{table}[H]
        \centering
        \includeAccuracyTableCore{#1}{#4}
        \caption{#3}
        \label{#2}
    \end{table}
}



\newcommand{\ErrorAnalysisTableBegin}
{
    \begin{table}[H]
    \centering
    \begin{tabular}{ll}
    \toprule
    \textbf{Model output} & \textbf{Correct output} \\ \midrule
}

\newcommand{\ErrorAnalysisTableRule}{\midrule}

\newcommand{\ErrorAnalysisTableEnd}[2]
{
\bottomrule
\end{tabular}
\caption{#2}
\label{#1}
\end{table}
}




% arguments: pdf file, label, caption
\newcommand{\includePDFPlot}[3]{
    \begin{figure}[H]
        \centering
        \includegraphics{#1}
        \caption{#3}
        \label{#2}
    \end{figure}
}

%%% document content
\begin{document}

\newcommand{\modIn}{{x}}
\newcommand{\modOut}{{\mathbf{y}}}
\newcommand{\modTarget}{{\hat{\mathbf{y}}}}
\newcommand{\modLayer}{{l}}
\newcommand{\modHiddenAt}[1]{{\mathbf{h}^{(#1)}}}
\newcommand{\nPos}{{n_{pos}}}
\newcommand{\dVocab}{{d_{vocab}}}
\newcommand{\dHidden}{{d_{hidden}}}
\newcommand{\nBlocks}{{n_{blocks}}}
\newcommand{\nHeads}{{n_{heads}}}
\newcommand{\dHead}{{d_{head}}}


\newcommand{\embTok}{{\mathbf{E}_{tok}}}
\newcommand{\embPos}{{\mathbf{E}_{pos}}}
\newcommand{\embTokAt}[1]{{\embTok[#1]}}
\newcommand{\embPosAt}[1]{{\embPos[#1]}}

%\def\WorkTitle{Transformer networks can learn arithmetic}
\def\WorkAuthor{Hugo Platzer}
\def\WorkAuthorMail{hugo.platzer@stud.plus.ac.at}
\def\WorkSupervisor{Helmut Mayer}
\def\WorkSupervisorMail{helmut@cs.sbg.ac.at}
\title{\WorkTitle}
\author{\WorkAuthor \\ University of Salzburg, Austria}

\begin{titlepage}
\centering \Large 

Department of Artificial Intelligence and Human Interfaces

University of Salzburg



\vspace{2cm} \LARGE
\WorkTitle

\vspace{1cm} \Large
\today

\normalsize
\vfill
\begin{tabular}{ll}
  \begin{tabular}{l}
  \bf Author: \vspace{0.1cm} \\ 
  \WorkAuthor \\
  \href{mailto:\WorkAuthorMail}{\WorkAuthorMail} \\
  \end{tabular}
  &
  \begin{tabular}{l}
  \bf Academic Supervisor: \vspace{0.1cm} \\ 
  \WorkSupervisor \\
  \href{mailto:\WorkSupervisorMail}{\WorkSupervisorMail} \\
  \end{tabular}
  \vspace{1cm} \\
  
  \begin{tabular}{l}
  \bf Correspondence to: \vspace{0.1cm} \\ 
  Universität Salzburg \\
  Fachbereich Artificial Intelligence and Human Interfaces \\
  Jakob--Haringer--Straße 2 \\
  A--5020 Salzburg \\
  \end{tabular}
  & \\
\end{tabular}

\vspace{1cm}

\end{titlepage}

% ensure title page is counted, makes PDF page numbers match Latex page numbers
\addtocounter{page}{1}

%\thispagestyle{empty}

\section*{Abstract}

Transformer networks have proven to be successful in a number of domains, such as natural language translation, coding, and general question answering. Arithmetic tasks can be challenging for them however.

nanoGPT is a small transformer network which allows for efficient training. This is true because...
We train a small (around 10M parameters) transformer decoder model on integer arithmetic tasks such as addition, multiplication, and integer square root for various operand lengths and evaluate accuracy.

Training samples are generated using ...

Performance is measured using ...

Error samples are studied for patterns, then a number of different strategies for improving performance such as different sampling distributions, reversing digits, and step-by-step reasoning are tried. The internal workings of the model are inspected for patterns using attention heatmaps.

\clearpage

\tableofcontents
\clearpage

\section{Introduction}

In 2017, the influential paper "Attention Is All You Need" \cite{allyouneed} was published. It showed that a new language model architecture, which they called the "Transformer", outperformed all previous models on English-French and English-German translation tasks, while also having lower training cost \citepage{8}{allyouneed}. Soon thereafter, variations of the transformer architecture described in this paper became the state of the art in most natural language processing tasks. They were also found to be competitive with established neural network architectures in other domains such as image classification \citepage{6}{worth16words}.

OpenAI's release of the ChatGPT web service in late 2022 \cite{openai_chatgpt_2022} made Transformer networks a topic of relevance for the general public. This AI chat assistant gave natural and helpful answers, mimicking human dialogue to a degree not seen before; while showing strong skills in a wide array of domains such as computer science, history and creative writing \cite{Savelka_2023} \citepage{5}{openai2023gpt4}.

While the details of the architecture of ChatGPT have not been released \cite{openai_chatgpt_2022} \cite{openai2023gpt4}, it is assumed that it is based on a transformer decoder architecture very similar to that of GPT2, which was released by OpenAI in 2019, with the main differences being that ChatGPT is based on a bigger model with more learnable parameters as well as a larger training dataset \cite{OpenGenus2023GPTComparison}.

Even though transformer models are very strong at learning the patterns of human language, they do have limitations in other areas such as performing arithmetic.
For example, ChatGPT is unable to correctly multiply numbers with 4 or more decimal digits. It does however seem capable of adding numbers of arbitrary size \cite{openai2023gpt4} \citepage{9}{nocalculator}.  This work seeks to explore the limits on what transformer-style models are able to learn in terms of integer arithmetic.

\subsection{Neural networks and arithmetic}

 Basic feed-forward networks can do addition of binary numbers (each input / output neuron representing one bit) with one hidden layer and binary multiplication with two hidden layers with perfect accuracy \cite{solving}. They did not train a network, but instead showed this is possible by constructing a feed-forward neural network including weight and bias matrices and proving its output always matches the sum or product of the input numbers.

\cite{visual} trained feed-forward neural networks on integer addition and multiplication, experimenting both with one-hot decimal digit vectors as well as 2D binary images of the written numbers as inputs and outputs. For addition, input integers were limited so the sum fits in 7 digits. For multiplication operands were limited so the product fits in 7 digits, thus allowing operands in the range $[0, 3162]$. Error rates achieved were low for addition ($\approx 2\%$ for both for vector and image inputs), but high for multiplication ($\approx 38\%$ for vector inputs and $\approx 72\%$ for image inputs).

\subsection{Transformer networks and arithmetic}

The main inspiration for our work was an experiment \cite{teaching}  to teach arithmetic to small transformer networks ("nanoGPT" with about 10 million parameters). First, they train for 3-digit integer addition and compare final model performance when they used various training dataset sizes. They then experiment with techniques like result reversal (reversing the order of output digits so the least significant digit comes first, this matches order of computing digits when adding by hand) and step-by-step thinking (not just having inputs and their sum in each training sample, but also intermediate steps of the computation, like intermediate sums and carry amounts, similar to adding or multiplying by hand) to improve performance when training with larger operands. They also experiment with 2-digit multiplication and approximating the square root of decimal numbers in the range $[0,10]$.

A much larger transformer network with about 7 billion parameters is trained in \cite{goat}. They try to train on addition and multiplication up to 16 digits while using step by step thinking (including intermediate results of each computation into the training sample, similar to doing addition / multiplication by hand). Accuracy was near-perfect on addition but they failed to learn multiplication.

Training transformer models on binary addition and multiplication was attempted in \cite{memtocomp}. This stands in contrast to the base 10 digits used in the papers mentioned before. They train on operands of only 7 bits and use almost all possible $2^7 \cdot 2^7 = 2^{14}$ addition or multiplication examples in the training set, with the remaining few reserved for the test set. Training eventually achieved perfect accuracy for addition and multiplication, although the operand sizes were small for this study.

 Length generalization is the capability of the model to work on operand lengths larger than the ones seen in training. This is the focus of \cite{rightembeddings}. They use a large training dataset of 20 million samples of mixed operand lengths. Using some tweaks to the model's embeddings, they achieve near perfect accuracy for addition and multiplication, as well as some degree of length generalization for addition.

Various transformer models from 10M to 2B parameters are trained on arithmetic tasks in \cite{nocalculator}. They divide training into two phases, with the first phase containing operations with operands up to 5 digits and the second phase containing operations with operands up to 12 digits. They achieve over 99\% accuracy for 3 and 5 digit addition for all model sizes. For 3 digit multiplication, accuracy obtained is between 78\% (10M parameters model) and 98\% (2B parameters model), with larger models achieving better results.
For 5 digit multiplication, accuracy obtained is between 42\% (10M parameters model) and 90\% (2B parameters model), with larger models achieving better results.

The idea of training a model to perform arithmetic computations step by step (meaning the training samples also contain intermediate results like when computing the result by hand) is explored in \cite{implicit}. After training the model on the step by step samples, they then train the model to no longer output the intermediate steps, just the final result. This then yields a model that just outputs the correct result for an arithmetic task.

In addition to the input token sequence, transformer networks usually also receive an additional position vector for each token that helps them know the position of a token in the sequence (since their learnable parameters are independent of token position).
Positional encoding schemes are ways in which positional information is embedded into the token sequence to make the model grasp the order of tokens. This ranges from no positional encoding at all to more sophisticated schemes involving vector rotation.
Various positional encoding schemes are compared for performance in  \cite{positionmatters}, with a focus on learning arithmetic and length generalization. 

The topic of length generalization is also explored in \cite{lengthandcount},  in the size of operands as well as the number of operands (e.g., multiple additions or multiplications per prompt).

Length generalization for the operations of addition, multiplication and modular arithmetic is also studied in \cite{lengthgen}. They tested various positional encoding schemes, and achieve perfect accuracy for addition from 6-20 digits for all schemes. For multiplication they trained on operations where the smaller operand has 3 decimal digits and the larger operand has 5 decimal digits. Then they tried to see how well this trained model performs when the larger operand has no longer 5, but up to 35 digits (length generalization). Multiplication with two large operands of 5 or 10 digits each was not tested.

%\section{Transformer models}

Machine translation, which is the automated translation from one natural language to another, was an important challenge since the beginning for the field of artificial intelligence and machine learning.
The input and output texts are usually modeled as sequences of integers from a limited range, with each integer representing a phrase, word, or part of a word.
Classical dense/feed-forward neural networks are a poor fit for such tasks, as the sequence lengths of even a short text combined with the many possible choices for each position make for a dense network with far too many parameters.

Thus researchers started using approaches like recurrent neural networks (RNN) \cite{phrasereps}, convolutional neural networks (CNN) \cite{convseq} and long short-term memory (LSTM) \cite{seq2seq}.
While these approaches did make the parameter count more manageable due to their sparse architecture, they still had issues with training on long sequences and learning long-distance correlations within a sequence.

Transformer models were introduced in the influential paper "Attention as all you need" \cite{allyouneed} in the context of machine translation to help address the aforementioned issues. Later on, they proved useful for many other tasks. 

Due to their parallelized architecture which will be explained in detail later, transformers allow for training on long sequences in a single optimization step, greatly increasing training throughput \citepage{1}{allyouneed}. They also have full connection between any two points in the sequence in each layer of the network, mitigating the problem of long-range dependencies being only considered across multiple model steps/layers in CNNs or RNNs which makes information loss more likely \citepage{6}{allyouneed}.

\subsection{Types of transformer models}

There are a variety of different albeit similar structures that have been used to build effective transformer networks. In particular, decoder, encoder-decoder, and encoder models need to be differentiated.

Encoder models such as BERT \cite{devlin2019bert} transform any given text into a vector of fixed size. This vector in the ideal case captures the essence of the text's meaning and can thus be used for tasks like text classification or sentiment analysis.

Encoder-decoder models like the original transformer architecture as described in \cite{allyouneed} use the vector generated by the encoder to start generating new text. This can be used for example by translation models or for answering questions.

Decoder models like the GPT models \cite{improvinglu} generate new text to extend some given starting text. This can be used for example to write a story when given a first paragraph. But with the right training, they can be used for the same tasks as encoder and encoder-decoder models as can be seen in the next section.

\subsection{Transformer decoder models}

The purpose of a decoder transformer model is to have it complete a sequence of tokens. These tokens may represent a letter of some alphabet, fragments of words, entire words, keywords of a programming language, even decimal digits, byte values or the binary values 0 and 1.
They could even represent color values or grid coordinates. Natural language modeling, however, is the most common scenario.

The model can be trained on token sequences generated from natural language taken from books, news articles, online communication platforms or other sources. This would make the model useful for writing stories, filling gaps in text and so on.
Or it could be trained on pieces of source code, which would allow the model to act as a code completion assistant.
The model would be given a piece of text that starts a story the user wants to have completed, or a piece of code that only has a function defined with comments. A well-trained model would then complete the code or story in a useful way so the rest of the story matches the theme set in the starting paragraph, or the code generated fits the comments and function signature written before.

The training texts could also all be in a structured question-answer pair form, allowing the model to answer questions. A few examples:
\label{transformer:examples}

\begin{itemize}
\item Translation tasks: "en: The house is big.  de: Das Haus ist groß."
\item Classification tasks: "text:  what a cool product, would buy again  sentiment: happy"
\item Mathematical tasks: "21*2=42" or "123: odd number" / "124: even number" \citepage{3}{nocalculator}
\end{itemize}

When the models is well-trained on these question-answer pair texts, it can be asked a question by simply leaving out the answer part at the end of the text, like "21*3=", as demonstrated in \cite{alammar-transformer}. A good model would be able to generate the correct answer for similar questions by inference from the many examples it has seen, even when the exact question posed is not in its training data.

\subsection{Beyond text}

Apart from text-based tasks, transformer models are also capable in domains like image classification.
A 2D image can be flattened into a sequence of tokens / vectors which can then be processed in a normal transformer model. The resulting image classifier achieves similar results to state-of-the-art CNN-based classifiers \citepage{3,5}{worth16words}.

Transformers can even be successful at generating images from text prompts. The approach presented in \cite{cogview} involves using a regular tokenizer for the text prompts and a separate discrete autoencoder to transform the training images into tokens and the generated tokens back to images.

Board games such as chess and go can also easily be tackled by transformers, as the board state and moves made can be transformed into text or represented by specialized tokenization schemes.
In \cite{grandmasterlevelchess} a transformer decoder model is trained to generate best moves or score board positions based on a large dataset of human chess games. The resulting chess playing engine manages grandmaster-level performance in blitz chess against human opponents \citepage{5}{grandmasterlevelchess}.

There have been minor modifications to the model architecture, data representation, and training regime among the diverse tasks outlined above. Also the performance of the resulting models is generally good but not quite state-of-the art. Nonetheless, having one model architecture that can achieve strong performance across a diverse set of domains seems promising as a step towards artificial general intelligence.


\subsection{Tokenization}
\label{transformer:tokenization}

Most transformer models preprocess input text using a so-called tokenizer, which is a model that was trained to group text into meaningful chunks out of a limited vocabulary called tokens to provide for an efficient representation of language \cite{subwordunits}. The tokenizer transforms the text into a sequence of integers, which is then processed by the transformer model which outputs token IDs which are then converted back into text.

Compared to just using raw bytes as model inputs, a good tokenizer allows for shorter sequence lengths (as a token typically represents multiple characters/bytes), meaning faster generation and greater maximum length \cite{tokenizerchoice}. Nonetheless, there are capable models such as Google ByT5 \cite{xue2022byt5} that do not need a tokenizer as they operate directly on byte values.
Compared to manually specifying a vocabulary (for example contents of a dictionary) tokenization is more flexible, having no issues with unknown words or large composite words (any possible byte sequence can still be represented) and is adaptable to the text corpus the model should be used on.

To train a tokenizer, the byte-pair-encoding (BPE) algorithm \cite{subwordunits} is often used, which also has applications in data compression. It works as follows:

\begin{enumerate}
	\item At the start, the set of token strings is just the possible characters in the input text, e.g., all ASCII characters. At this point, each character needs one token (string) in the representation of the tokenizer.
	\item The input text is modeled as a sequence of token strings from the token strings set, preferring longer strings wherever possible over shorter ones. In this sequence, the most common adjacent pair of two token strings is identified.
	\item A new token string is created from the concatenation of this pair, e.g., if "A" followed by "B" is the most common token string pair in the text, a new token string "AB" is added to the set.
	\item Steps 2 and 3 are repeated over and over. This causes the token vocabulary to grow and the average number of tokens needed per character to decrease. This is done until a good balance between token vocabulary size and tokens per character is achieved.
\end{enumerate}

Apart from the regular token vocabulary, most tokenizers also contain some special token values that will not be found in the input text. These are used for example to represent the start and end of each training sample, delimit individual pieces of text when they are fed to the model, or separate parts of a question/answer pair \cite{HuggingFaceTokenizers}.


\subsection{Input / output}
\label{transformer:input}

Formally, the input to the model consists of a sequence of integers (token IDs) $\modIn_1 \ldots \modIn_n$, where $n$ is the length of the token sequence.
Generally, transformer models place a limit on the length of the sequence $n$ due to their internal structure, we denote this limit as $\nPos$. However, models with no such limitation have also been proposed \cite{su2023roformer}.
Each integer $\modIn_i$ references one of the possible tokens of the model $(1 \leq \modIn_i \leq \dVocab)$. $\dVocab$ is the size of the token vocabulary, which depends on the tokenization process used for the model \citepage{2}{allyouneed}.

\label{transformer:output}

For each position in the input sequence, the model outputs a probability distribution, predicting the next token based on the input tokens up to this position. 
The output of the model is a sequence of probability distributions $\modOut_i$ $(1 \leq i \leq n)$. $\modOut_i \in \mathbb{R}^\dVocab$, $0 \leq (\modOut_i)_j \leq 1$ \citepage{5}{allyouneed} \cite{HuggingFaceGPT2}.
$\modOut_i$ should model which token most likely follows after all the tokens $\modIn_1 \ldots \modIn_n$.

During training, all output positions are useful as they allow to optimize the model to correctly predict $\modIn_2$ based on $\modIn_1$,
$\modIn_3$ based on $\modIn_1$ and $\modIn_2$, $\modIn_n$ based on $\modIn_1 \ldots \modIn_{n-1}$, all in a single training step. This greatly reduces training time for longer sequences \cite{alammar-gpt2}.

\subsection{Text generation}
\label{transformer:textgen}

When using the trained model to generate text, only the last output distribution, $\modOut_n$, is of interest, because
it tells the next token after the given input sequence. \cite{Mao2021Autoregressive}

This is used to generate one new token at a time (autoregressive decoding). The new token is picked from the distribution either by choosing the most likely token (greedy decoding), random sampling according to the probability distribution, or more sophisticated methods such as top-p sampling (choosing among the most likely tokens whose probability mass equals for example $p=0.95$) and top-k sampling (selecting from the for example $k=10$ most likely tokens) \cite{ippolito2019comparison}.

The newly generated token $\modIn_{n+1}$ is appended to the input sequence and the whole process is repeated, with $n$ growing by one after each step, until a special end-of-sequence token is generated, the maximum length $\nPos$ is reached, or some other stop condition is triggered \cite[sect. "Generation"]{HuggingFaceGPT2}. If generation is not terminated, the model could in theory generate a text of infinite length, although transformer models generally only have a limited window of past tokens they can consider when generating the next token. In addition, generation time and memory requirements generally increase quadratically with sequence length for transformer models (\cref{gpt2:attn}).

\subsection{User interaction}

In 2023, OpenAI released their most successful product yet, ChatGPT \cite{openai_chatgpt_2022}. This online service allows users to chat with an artificial intelligence assistant which can help with reciting facts, writing texts and code, giving advice of all sorts and so on.

The principle by which transformer decoder models can act as chat assistants is simple: The model is trained on conversations like this:

\begin{verbatim}
    User: What color is the sky?
    Assistant: The sky is blue.
\end{verbatim}

The model can then become the helpful assistant by filling in the assistant's messages:

\begin{verbatim}
    User: What color is a banana?
    Assistant:
\end{verbatim}

and the model would generate a response like "A banana is yellow" followed by an end-of-sequence token, similar to its training examples. For more details on how generic language models can be tuned to follow instructions or conversation, see \cite{yi2019coherent} and \cite{ouyang2022training}.

\subsection{Generative pre-training}

The GPT2/GPT3/ChatGPT transformer models have demonstrated the effectiveness of a technique used to train language models called generative pre-training (GPT). This technique was first popularized in \cite{improvinglu} and consists of two phases.

In the pre-training stage the model is first trained on a large corpus of text in multiple languages, including websites, books, articles, chats and source code. In this manner, a huge amount of training data can be collected without the need for careful manual selection / preparation of question-answer pairs. The training data is split into appropriate chunks and the model learns to predict (generate) the next word based on the previous ones for each chunk as a form of unsupervised learning \citepage{3}{improvinglu}.

After sufficient pre-training, the model is then fine-tuned on question-answer pairs of the specific task the model should be able to solve, like the examples presented in \cref{transformer:examples} or even human-assistant chats on a wide range of topics.

Due to the well-known phenomenon of transfer learning \cite{transferlearning}, the pre-trained model will have already internalized the structure of natural language, source code and math. 
This allows a relatively small amount of hand picked fine-tuning examples to make it significantly stronger compared to just training on the question-answer pairs on a freshly initialized model \citepage{8}{improvinglu}.

\subsection{Attention mechanism}
\label{transformer:attn}

Before transformers, model architectures such as recurrent neural networks (RNN) and Long Short Term Memory (LSTM) were commonly used to handle textual or other sequential data. They ingest the tokenized text one token at a time, keeping one or multiple fixed vectors of model state. Transformers, in contrast, dropped recurrence and handle the entire sequence in parallel \citepage{1}{allyouneed}.

To draw connections between elements of the token sequence, they use a so-called attention mechanism. It is in fact the only part of the model that connects the different parts of the sequence it is processing, all other components of a transformer model handle each position independently. The concept of an attention mechanism is best explained using a high-level example \cite{alammar-gpt2}: Suppose the model is processing the sentence "A robot must obey orders given to it" and each word is one token. The model should learn that there is a connection from the pronoun "it" to the earlier noun "robot".
The model's attention mechanism would do the following when handling the token "it":
\begin{enumerate}
	\item Generate a query vector $\mathbf{q}_{\mathrm{it}} = f_q("\mathrm{it}")$ for the token "it" using a function $f_q$ that depends on learned parameters of the model (often a linear transformation).
	\item Generate key vectors $\mathbf{k}_t$ for all the tokens $t$ of the sentence, using another learned function $f_k$: $\mathbf{k}_{\mathrm{a}} = f_k(\mathrm{"a"})$, $\mathbf{k}_{\mathrm{robot}} = f_k(\mathrm{"robot"})$, \ldots, $\mathbf{k}_{\mathrm{it}} = f_k(\mathrm{"it")}$.
	\item Compute numeric similarity scores $s$ between the query vector and each key vector, using a measure like dot product: $s_{\mathrm{a}} = \mathbf{\mathbf{q}}_{\mathrm{it}} \cdot \mathbf{k}_{\mathrm{a}}$, $s_{\mathrm{robot}} = \mathbf{q}_{\mathrm{it}} \cdot \mathbf{k}_{\mathrm{robot}}$, \ldots, $s_{\mathrm{it}} = \mathbf{q}_{\mathrm{it}} \cdot \mathbf{k}_{\mathrm{it}}$.
	\item Normalize the similarity scores $s$ using softmax to get the normalized scores $s'_t$ for each token $t$. This ensures that $\sum_t s'_t = 1$.
	\item Generate value vectors $\mathbf{v}_t$ for all the tokens $t$ of the sentence, using another learned function $f_v$: $\mathbf{v}_{\mathrm{a}} = f_v(\mathrm{"a"})$, $\mathbf{v}_{\mathrm{robot}} = f_v(\mathrm{"robot"})$, \ldots, $\mathbf{v}_{\mathrm{it}} = f_v(\mathrm{"it")}$.
	\item Using the normalized similarity scores, blend the value vectors together to get the attention output vector for "it": $\mathbf{a}_{\mathrm{it}} = \sum_t s'_t \mathbf{v}_t$. 
\end{enumerate}

Assuming the model has learned to generate meaningful query and key vectors, the similarity score to the query of "it" should be much higher for the key of "robot" than for the other positions' keys. This would cause the attention output vector of "it" to mostly consist of the value vector of the token "robot", with the other positions' value vectors having little impact. Thus data flows to the position of the token "it" mostly from the position of the token "robot", which models the linguistic connection between these words.

It has to be said however that this explanation has been simplified considerably. Transformer models do not operate on words or tokens directly but first transform them into vectors of fixed size that are then fed through the attention mechanism. There are multiple stacked blocks inside a typical transformer model, each consisting of an attention mechanism among other components. A more precise description of the attention mechanism of a realistic transformer model is given in \cref{gpt2:attn}.

\subsection{OpenAI GPT2}
\label{gpt2}

A good example of a successful decoder model is OpenAI's GPT2 \cite{unsupervisedmultitask} \cite{OpenAI2019BetterLM} \cite[source code]{HuggingFaceGPT2}. It was released in 2019 and demonstrated the ability to write realistic articles based on a given starting prompt. This model is useful for explaining the details of the transformer architecture, because the model weights and code were released, its proven effectiveness in generating meaningful text, and due to the fact that its even more powerful successors have not experienced significant architectural changes \cite{OpenGenus2023GPTComparison}.

\subsubsection{Achitecture}

A general overview of the architecture of GPT2 is presented in \cref{diagrams/gpt2/overview}.
As described in \cref{transformer:input}, the input to the model is an integer sequence, with each integer representing an element of the token vocabulary.
Creating this token sequence from the input text is done by the tokenizer (\cref{transformer:tokenization}, \cref{gpt2:tokenizer}). This tokenizer is considered a component separate from the model and is trained before it. The model is then trained based on the output indices of the chosen tokenizer, therefore the tokenizer cannot be changed for a trained model. The various stages inside the model are as follows:

\begin{enumerate}
	\item The sequence of token indices is then transformed into a sequence of vectors by the preprocessing stage (\cref{gpt2:preproc}). The preprocessing stage converts each token $\modIn_i \in \mathbb{N}$ into a hidden vector $\modHiddenAt{0}_i \in \mathbb{R}^\dHidden$ based on the token value $x_i$ and token position $i$. For this, it uses learnable embedding matrices.
	The size of each of these hidden vectors, $\dHidden$ is a parameter that can be configured when creating a GPT2 model (\cref{gpt2:parameters}).
	
	\item These vectors $\modHiddenAt{0}_i$ are then fed through a number of GPT2 blocks, each being identical except for having its own set of learnable parameters. These GPT2 blocks are indexed by $l$ ($1 \leq l \leq \nBlocks)$, with $\nBlocks$ being the number of GPT2 blocks the model has, which can also be configured at model creation.
	We denote the input vectors to block number $\modLayer$ as $\modHiddenAt{\modLayer-1}_i$ and the output vectors of that block as $\modHiddenAt{\modLayer}_i$. Thus for a model with, e.g., 12 blocks, the inputs to the first block are $\modHiddenAt{0}_i$ and the outputs after the last block are $\modHiddenAt{12}_i$.
	
	Each block outputs a vector sequence of the same dimensions as its input.
	The GPT2 block structure is described in detail in \cref{gpt2:block} and consists of multiple layers including normalization, a feed-forward network and the attention mechanism. The attention mechanism (\cref{gpt2:attn}) is special to transformer networks and the only point in the model where vectors are not transformed independently, i.e., the output vector of the attention mechanism at position $i$  depends not just on the input vector at position $i$, but also on the vectors at all previous positions $1 \leq i' < i$.
	The job of the block and the attention mechanism in particular is to allow the model to learn how later elements of the training sequences are influenced by previous elements, e.g., capitalized words mostly appearing after points and spaces (ends of sentences) in English training texts.
	
	\item After the last block there is the postprocessing stage (\cref{gpt2:postproc}). Its job is to convert each hidden vector $\modHiddenAt{\nBlocks}_i$ into a probability distribution $\modOut_i \in \mathbb{R}^{\dVocab}$. This is also done with a learnable embedding matrix.
	
	During training (\cref{gpt2:training}), the model parameters are optimized so that every $\modOut_i$ correctly predicts the next token in the training sequence, $\modIn_{i+1}$.
	During inference (\cref{gpt2:textgen}), only the last distribution, $\modOut_n$, is used to sample a predicted token $\modIn_{n+1}$, which then is appended to the input sequence for the next generation step.
\end{enumerate}

\includediagram{diagrams/gpt2/overview}{Overview of a GPT2 model. This example uses a model with $\nBlocks=12$ blocks, $\dHidden=768$ hidden vector size. The model is processing a sequence of length 3. Models can also process longer sequences than shown in this example, up to their positional limit $\nPos$ which is configurable at model creation time as is the number of blocks and hidden vector size.}



\subsubsection{Model size variants}

\label{gpt2:parameters}

To make the model bigger (needing more memory and computation time, but allowing for more accurate generation) or smaller (faster, but lower quality output), two hyperparameters can be adjusted \citepage{4}{unsupervisedmultitask} \citepage{5}{improvinglu}: The vector size for each position can be changed just like the number of transformer blocks. The number of attention heads is usually adapted so that the size of each head remains around 64 \citepage{4}{unsupervisedmultitask} \cite{hfpretrained}.
OpenAI GPT2 was released in four versions with different parameter counts:

\includeTable
{
	name; n_params; n_blocks; d_hidden; n_heads; d_head\\
	gpt2-small; 117M; 12; 768; 12; 64 \\
	gpt2-medium; 345M; 24; 1024; 16; 64 \\
	gpt2-large; 774M; 36; 1280; 20; 64 \\
	gpt2-xl; 1558M; 48; 1600; 25; 64 \\
}
{tab:gpt2-sizesX}
{
	Versions of GPT2 released with different parameter counts \cite{hfpretrained}: $\nBlocks$ is the number of GPT2 blocks, $\dHidden$ is the vector size at each position, $\nHeads$ is the number of attention heads and $\dHead$ is the size of each attention head.
}
{%
	columns/name/.style={column name={Model variant}},
	columns/n_params/.style={column name={No. of parameters}},
	columns/n_blocks/.style={column name={$\nBlocks$}},
	columns/d_hidden/.style={column name={$\dHidden$}},
	columns/n_heads/.style={column name={$\nHeads$}},
	columns/d_head/.style={column name={$\dHead$}}
}


\subsubsection{Tokenization}

\label{gpt2:tokenizer}

The GPT2 models as released by OpenAI were trained using a Byte-pair encoding (BPE) tokenizer (\cref{transformer:tokenization}) with a vocabulary of 50257 tokens \citepage{4}{unsupervisedmultitask} \cite{HuggingFaceGPT2}. The tokenizer was itself trained on a natural language corpus separately before training the actual models. The models were then configured with $\dVocab=50257$ and the training texts were processed using this tokenizer before being fed into the model. Thus, when using these models with the weights as released by OpenAI, the same tokenizer must be used for text generation during inference time as well. An example on how the tokenizer that was released with OpenAI GPT2 handles a sentence can be seen in \cref{diagrams/gpt2/tokenizer}.

It is also possible to train a GPT2 model with a different tokenizer with, e.g., a smaller vocabulary. Instead of a proper tokenizer that was trained to efficiently chunk text in some natural language, other GPT2 implementations like nanoGPT \cite{nanogpt} use a simpler approach: Each ASCII character in the training text is mapped to one token, with the token value equal to the ASCII character value.
This makes for a simpler overall implementation that keeps all the processing outside the tokenizer and inside the actual model.
The downside is that token sequences get significantly longer, thus a model with the same $\nPos$ and ASCII tokenization can only handle significantly shorter texts than one with a trained tokenizer that recognizes, e.g., frequent words and groups all their characters into a single token.

\includediagram{diagrams/gpt2/tokenizer}{Operation of the GPT2 tokenizer on an example sentence.}


\subsubsection{Preprocessing}
\label{gpt2:preproc}

Transformer networks do not operate on integer token IDs, but on state vectors from $\mathbb{R}^{\dHidden}$. $\dHidden$, the dimension of the state vector at each position, is a parameter fixed at the time of model construction (\cref{gpt2:parameters}).
This means the token sequence needs to be converted into a sequence of such vectors before feeding it through the model. This happens during the preprocessing stage, which is illustrated in \cref{diagrams/gpt2/preproc}.
For converting the token IDs $\modIn_i$ into initial state vectors $\modHiddenAt{0}_i$, GPT2 has a learnable token embedding matrix $\embTok \in \mathbb{R}^{\dVocab \times \dHidden}$. Each row of this matrix stores the learnable initial state vector for the corresponding token ID.


Because the blocks of the GPT2 have no positional awareness (their learnable parameters do not depend on the sequence position $i$), information about the position of a token within the sequence needs to be included as well \citepage{6}{allyouneed}.
In GPT2, this is done by a learnable embedding matrix $\embPos \in \mathbb{R}^{\dVocab \times \dHidden}$ storing a learnable vector for each position up to the sequence length limit $\nPos$.

The starting vector $\modHiddenAt{0}_i \in \mathbb{R}^\dHidden$ for a token value $\modIn_i$ at position $i$ is then given in \cref{eq:gpt2_preproc} as the sum of the token and positional embedding vectors \citepage{3}{allyouneed}:

\begin{equation}
	\modHiddenAt{0}_i = \embTokAt{\modIn_i, :} + \embPosAt{i, :}
	\label{eq:gpt2_preproc}
\end{equation}

\includediagram{diagrams/gpt2/preproc}{GPT2 preprocessing stage on an example token sequence. Based on the token position and value, embedding vectors are summed to give the starting vector at this position.}


\subsubsection{Postprocessing}
\label{gpt2:postproc}

GPT2 uses a "language modeling head" \cite{HuggingFaceGPT2} to convert the outputs of the last block $\modHiddenAt{\nBlocks}_i$ into token probabilities $\modOut_i$, the steps of which are illustrated in \cref{diagrams/gpt2/postproc}.

First, a final layer normalization step is performed \citepage{4}{unsupervisedmultitask}, which transforms $\modHiddenAt{\nBlocks}_i$ into $\modHiddenAt{\nBlocks, norm}_i$. The details of this layer normalization step are  the same as the layer normalization steps in each GPT2 block, which are described in \cref{gpt2:layernorm}.

GPT2 then re-uses the same matrix $\embTok$ that was used in the preprocessing stage to convert $\modHiddenAt{\nBlocks, norm}_i$ into a token probability vector $\modOut'_i$ of dimension $\dVocab$, using matrix-vector product, as shown in \cref{eq:postproc} and \cref{eq:postproc_detailed}.
This tying of the weights in the postprocessing stage to those of the preprocessing stage is done for improving model performance as well as reducing the number of learnable parameters \cite{weightstying}.

Finally, softmax is applied to transform this vector $\modOut'_i$ into a proper probability distribution $\modOut_i$.

\begin{align}
	\modOut'_i &= \embPos \modHiddenAt{\nBlocks, \textrm{norm}}_i
	\label{eq:postproc}
	\\
	(\modOut'_i)_k &= \sum_{j=1}^{\dHidden}{\embPosAt{k,j} (\modHiddenAt{\nBlocks, \textrm{norm}}_i)_j} & 1 \leq k \leq \dVocab
	\label{eq:postproc_detailed}
\end{align}

\includediagram{diagrams/gpt2/postproc}{Detailed example of the GPT2 postprocessing stage. $\modHiddenAt{\nBlocks}_i$, the output of the last GPT2 block, is converted into token probabilities.}


\subsubsection{Text generation}
\label{gpt2:textgen}

Generating output text using a trained GPT2 model follows the general process for transformer networks outlined in \cref{transformer:textgen}.
The generation process is visualized in \cref{diagrams/gpt2/generation}. It always starts with some existing starting text that is transformed into a sequence of $n$ tokens $\modIn_1 \ldots \modIn_n$. These tokens are fed through the model to get the output probability distributions $\modOut_1 \ldots \modOut_n$. The last distribution $\modOut_n$ is sampled using one of the sampling methods outlined in \cref{transformer:textgen}. The token that results from this sampling becomes $\modIn_{n+1}$ and is appended to the sequence $\modIn_1 \ldots \modIn_n$. The process outlined above is then repeated, now sampling $\modOut_{n+1}$ to get the next token $\modIn_{n+2}$ and so on.

Depending on what texts / token sequences the model was trained on, it will at some point cause a special stop token ID to be sampled, which tells the sampling process that the generated text is now complete and sampling should be stopped.
In other cases, a fixed number of tokens defined by the user might be sampled, which is possible as long as the total token sequence length does not exceed $\nPos$.

Due to the limited dimensions of the positional embedding matrix $\embPos$, GPT2 as described here does not support generating texts of unlimited length. Other variants of GPT2 or similar transformer models use different positional embeddings not based on a matrix of fixed dimension. Those models could support unlimited generation, although it is limited in practice by computational complexity and limited coherence in the generated text (\cref{transformer:textgen}).

\includediagram{diagrams/gpt2/generation}{Text generation using the GPT2 model for the example starting text "where is". The rightmost probability distribution is used to pick the next token, after which the data, augmented with this new token, is fed through the model again.}

\subsubsection{Training}
\label{gpt2:training}

GPT2 uses a training objective similar to \citepage{3}{improvinglu} where a large text corpus is used to make the model predict the next token based on the tokens up to this point.
In \cite{improvinglu} this is called "unsupervised pre-training", however other authors refer to similar automated labeling techniques as "self-supervised learning" \cite{selfsupervised} \cite{selfsupervisedvisual}.
In contrast to classical supervised learning, the training samples in self-supervised learning are not manually labeled, but the inputs and labels are automatically created from each training sample using some mathematical function. An example of this could be erasing some pixels of an image in the input while leaving them untouched in the label.
For training transformer decoder models like GPT2, the training goal typically is to make the model correctly complete a text similar to the training samples when only the start of such a text is given.
To achieve this, each training sample's token sequence $\modIn_1 \ldots \modIn_n$ is shifted by one position: the inputs to the model are $\modIn_1 \ldots \modIn_{n-1}$ and the corresponding labels are $\modIn_2 \ldots \modIn_n$. This makes the model learn to predict $\modIn_{i+1}$ based on $\modIn_1 \ldots \modIn_i$ for all $i$ from $1$ to $n-1$ in a single optimization step (also see \cref{transformer:output}).

For each probability distribution $\modOut_i$ output by the model, cross-entropy loss is used to measure the deviation between $\modOut_i$ and the ideal $\modTarget_i$,
where $\modTarget_i$ is the probability distribution of a guaranteed token with ID $\modIn_{i+1}$ \citepage{2}{unsupervisedmultitask} \cite{HuggingFaceGPT2}.

Standard initialization techniques (for example uniform initialization) can be used for the weight and bias matrices \citepage{5}{improvinglu}.

GPT2 is trained using the Adaptive Moment Optimizer (Adam)  \citepage{5}{improvinglu}.
A naive stochastic gradient optimizer will not achieve good results when training transformer networks like GPT2, thus more advanced optimizers like Adam need to be used to make the model converge to a degree where it becomes useful \cite{adambeatssgd}.






\subsection{GPT2 block structure}
\label{gpt2:block}

The GPT2 blocks are the core of the model, where it learns to draw connections between the vectors at different positions, and data can flow between positions.
Because its structure is somewhat complex, we dedicated this separate section to it.
A high-level overview of the structure of a GPT2 block is presented in \cref{diagrams/gpt2/block} as well as equations \eqref{eq:gpt2_overview_start} to \eqref{eq:gpt2_overview_end}:

\begin{enumerate}
	\item \Cref{eq:gpt2_overview_start} shows the input vectors to the block with index $\modLayer$, given as $\modHiddenAt{l, 0}_1 \ldots \modHiddenAt{l, 0}_n$, are the are the outputs of the previous block $\modHiddenAt{l-1}_1 \ldots \modHiddenAt{l-1}_n$. For the first block, its inputs are the outputs of the preprocessing stage (\cref{gpt2:preproc}).
	
	\item The first step of each block is a layer normalization stage, described by \cref{eq:gpt2_overview_ln1}. It is explained in more detail in \cref{gpt2:layernorm}.
	
	\item After the layer normalization stage the vectors pass through the heart of the block - the multi-head attention mechanism. \Cref{eq:gpt2_overview_attn} shows that multi-head attention, in contrast to the other stages, needs access to the entire vector sequence $\modHiddenAt{l, 1}_1 \ldots \modHiddenAt{l, 1}_n$ as well as the position $i$ for which to compute the attention output. This is necessary because it connects the vectors at different positions and lets data flow between them. Multi-head attention is explained in detail in \cref{gpt2:attn}.
	
	\item After the attention mechanism, there is a residual connection described by \cref{eq:gpt2_overview_res1}, which lets data flow around the layer norm and attention stages as well as through it. The motivation behind this is explained in \cref{gpt2:residual}.
	
	\item Then, there is a second layer normalization stage described by \cref{eq:gpt2_overview_ln2}. It works the same as the first layer normalization stage, but with separate learnable parameters. It is followed by the multi-layer perceptron (MLP) stage, described by \cref{eq:gpt2_overview_mlp}. This stage is a basic feed-forward network with one hidden layer, explained in in \cref{gpt2:mlp}.
	
	\item Finally, there is a second residual connection, described by \cref{eq:gpt2_overview_res2}, whose outputs as defined by \cref{eq:gpt2_overview_end} are the outputs of the block, $\modHiddenAt{l}_1 \ldots \modHiddenAt{l}_n$.
\end{enumerate}


The layer normalization, attention and MLP stages all have learnable parameters, with each GPT2 block having a separate set of such learnable parameters. All stages except the attention stage operate on each vector $\modHidden_i$ independently.


\begin{align}
	\modHiddenAt{l, 0}_i &= \modHiddenAt{l-1}_i
	\label{eq:gpt2_overview_start}
	\\
	\modHiddenAt{l, 1}_i &= \mathrm{LN1}\AtLayer{l}(\modHiddenAt{l, 0}_i)
	\label{eq:gpt2_overview_ln1}
	\\
	\modHiddenAt{l, 2}_i &= \mathrm{MHA}\AtLayer{l}(\modHiddenAt{l, 1}_1 \ldots \modHiddenAt{l, 1}_i)
	\label{eq:gpt2_overview_attn}
	\\
	\modHiddenAt{l, 3}_i &= \modHiddenAt{l, 0}_i + \modHiddenAt{l, 2}_i
	\label{eq:gpt2_overview_res1}
	\\
	\modHiddenAt{l, 4}_i &= \mathrm{LN2}\AtLayer{l}(\modHiddenAt{l, 3}_i)
	\label{eq:gpt2_overview_ln2}
	\\
	\modHiddenAt{l, 5}_i &= \mathrm{MLP}\AtLayer{l}(\modHiddenAt{l, 4}_i)
	\label{eq:gpt2_overview_mlp}
	\\
	\modHiddenAt{l, 6}_i &= \modHiddenAt{l, 3}_i + \modHiddenAt{l, 5}_i
	\label{eq:gpt2_overview_res2}
	\\
	\modHiddenAt{l}_i &= \modHiddenAt{l, 6}_i
	\label{eq:gpt2_overview_end}
\end{align}



\includediagram{diagrams/gpt2/block}{Overview of the components of a GPT2 block with index $\modLayer$. This example shows 3 positions $\modHiddenAt{\modLayer}_1 \ldots \modHiddenAt{\modLayer}_3$, but the actual model is flexible regarding sequence length.}

\subsubsection{Notation used}
\label{gpt2:notation}

To keep the diagrams and equations readable when describing the stages of the GPT2 block (layer norm, attention, MLP) next, we define the inputs to the stage simply as $\modHidden_i$ for each position $i$ from 1 to $n$, intermediate results as $\modHiddenStep{1}_i$ and the outputs as $\modHidden'_i$, dropping the block index $l$ and stage index.
Each block and each stage of the block has separate, independent learnable parameters despite this not being reflected in the notation used below.

\subsubsection{Layer normalization}
\label{gpt2:layernorm}

The two layer normalization stages of the GPT2 block are illustrated \cref{diagrams/gpt2/layernorm} as well as equations \eqref{eq:gpt2:ln_intermediate} to \eqref{eq:gpt2:ln_ln2}.

First, the input vector $\modHidden_i$ is normalized in \cref{eq:gpt2:ln_intermediate} by subtracting its mean $\mu$ and dividing by its standard deviation $\sigma$, with $\epsilon$ being a small positive constant added for numerical stability of the square root.

Then, a gain and bias value is applied per component to the intermediate result $\modHiddenStep{1}_i$. The first layer normalization stage LN1 uses learnable gain and bias vectors $\mathbf{w}_\mathrm{LN1}, \mathbf{b}_\mathrm{LN1} \in \mathbb{R}^\dHidden$ as described in \cref{eq:gpt2:ln_ln1}.
The second layer normalization stage LN2 uses separate gain and bias vectors $\mathbf{w}_\mathrm{LN2}, \mathbf{b}_\mathrm{LN2} \in \mathbb{R}^\dHidden$ as described in \cref{eq:gpt2:ln_ln2}.


\begin{align}
	(\modHiddenStep{1}_i)_j &= \frac
		{(\modHidden_i)_j - \mu(\modHidden_i)}
		{\sqrt {\sigma^2(\modHidden_i)+\epsilon}}
	\label{eq:gpt2:ln_intermediate}
	\\
	(\modHidden'_i)_j &= (\mathbf{w}_{\textrm{LN1}})_j (\modHiddenStep{1}_i)_j + (\mathbf{b}_{\textrm{LN1}})_j
	\label{eq:gpt2:ln_ln1}
	\\
	(\modHidden'_i)_j &= (\mathbf{w}_{\textrm{LN2}})_j (\modHiddenStep{1}_i)_j + (\mathbf{b}_{\textrm{LN2}})_j
	\label{eq:gpt2:ln_ln2}
\end{align}

\includediagram{diagrams/gpt2/layernorm}{Detailed view of the layer normalization stages of the GPT2 block. $d=\dHidden$. For the first layer norm LN1, $\mathbf{w}=\mathbf{w}_\mathrm{LN1}, \mathbf{b}=\mathbf{b}_\mathrm{LN1}$. For the second layer norm LN2, $\mathbf{w}=\mathbf{w}_\mathrm{LN2}, \mathbf{b}=\mathbf{b}_\mathrm{LN2}$.}

\subsubsection{Residual connection}
\label{gpt2:residual}

A residual connection, as described in \cref{eq:gpt2_res}, works by adding the inputs of some function $f$, like a layer in a machine learning model, back to its outputs \citepage{3}{allyouneed} \cite{residual}.
When using residual connections, the input data does not need to flow through all computation steps before reaching a given point, instead it can bypass some of them, allowing later computations easier access to earlier data.
Such connections are found in almost all models having many steps/layers, because they were found to greatly help when training deep models \citepage{1-2}{residual}.

\begin{align}
	\modHidden'_i = \modHidden_i + f(\modHidden_i)
	\label{eq:gpt2_res}
\end{align}


\subsubsection{Multi-head attention}
\label{gpt2:attn}

GPT2, just like the original transformer introduced in \cite{allyouneed}, uses a form of attention mechanism called multi-head attention \citepage{4}{unsupervisedmultitask}.\\
The stages of multi-head attention are presented in \cref{diagrams/gpt2/attention_multihead} and equations \eqref{eq:mha:q} to \eqref{eq:mha:output}.
At first, query, key and value vectors are computed for each position, which are then split into segments of equal size called attention heads. Then, the actual attention values are computed for each head, which are then merged and finally subjected to a linear projection \citepage{4}{allyouneed}.

The reasoning behind introducing multiple attention heads is given in \citepage{5}{allyouneed} as allowing the model to better attend to multiple separate positions $i'$ from some position $i$, each position being highlighted by a separate attention head. When just using a single attention head, blending the value vectors (see \cref{gpt2:attn_dot}) causes this information to become more diluted compared to having each $i'$ focused in a separate attention head's output and then concatenated.

Based on the input vectors $\modHidden_1 \ldots \modHidden_i$, the multi-head attention output $\modHidden'_i$ is computed as follows: 

\begin{enumerate}
	\item For each position $i' \leq i$, a query vector $\mathbf{q}_{i'}$, a key vector $\mathbf{k}_{i'}$, and a value vector $\mathbf{v}_{i'}$ are generated using linear transforms with learnable parameters $(\mathbf{W}_q, \mathbf{b}_q),(\mathbf{W}_k, \mathbf{b}_k), (\mathbf{W}_v, \mathbf{b}_v)$, as described by equations \eqref{eq:mha:q} to \eqref{eq:mha:v}. Each of these vectors has dimension $\dHidden$, just like $\modHidden_i$.
	
	\item These vectors are each split into $\nHeads$ segments of equal size $\dHead$. This splitting including the slicing indices $j_1, j_2$ is described by equations \eqref{eq:mha:qh} to \eqref{eq:mha:vh}. Each slice index $h$ represents an attention head.
	
	\item For each attention head $h$, scaled dot-product attention (\cref{gpt2:attn_dot}) is used to compute an attention output $\mathbf{a}_{i_h}$ at position i for that head, as described by \cref{eq:mha:sdpa}. This needs the query, key, and value vectors for that head at each position up to $i$.
	
	\item These attention outputs $\mathbf{a}_{i_h}$ are concatenated for all heads, giving a total attention output $\mathbf{a}_i$, as described by \cref{eq:mha:concat}.
	
	\item The concatenated attention outputs are passed through a linear transform with learnable parameters $(\mathbf{W}_{\mathrm{proj}}, \mathbf{v}_{\mathrm{proj}})$, as described by \cref{eq:mha:output}. This results in the output of the multi-head attention algorithm, $\modHidden'_i$.
\end{enumerate}

\begin{align}
\mathbf{q}_{i'} &= \mathbf{W}_q \modHidden_{i'} + \mathbf{b}_q
\quad 1 \leq i' \leq i
\label{eq:mha:q}
\\
\mathbf{k}_{i'} &= \mathbf{W}_k \modHidden_{i'} + \mathbf{b}_k
\quad 1 \leq i' \leq i
\label{eq:mha:k}
\\
\mathbf{v}_{i'} &= \mathbf{W}_v \modHidden_{i'} + \mathbf{b}_v
\quad 1 \leq i' \leq i
\label{eq:mha:v}
\\
\mathbf{q}_{i_h} &= (\mathbf{q}_i)_{j_1:j_2}
\quad 1 \leq h \leq \nHeads, j_1 = (h-1) \cdot \dHead + 1, j_2 = h \cdot \dHead
\label{eq:mha:qh}
\\
\mathbf{k}_{i_h} &= (\mathbf{k}_i)_{j_1:j_2}
\quad 1 \leq h \leq \nHeads, j_1 = (h-1) \cdot \dHead + 1, j_2 = h \cdot \dHead
\label{eq:mha:kh}
\\
\mathbf{v}_{i_h} &= (\mathbf{v}_i)_{j_1:j_2}
\quad 1 \leq h \leq \nHeads, j_1 = (h-1) \cdot \dHead + 1, j_2 = h \cdot \dHead
\label{eq:mha:vh}
\\
\mathbf{a}_{i_h} &= \operatorname{SDPA}(
	\mathbf{q}_{1_h} \ldots \mathbf{q}_{i_h},
	\mathbf{k}_{1_h} \ldots \mathbf{k}_{i_h},
	\mathbf{v}_{1_h} \ldots \mathbf{v}_{i_h}
)
\label{eq:mha:sdpa}
\\
\mathbf{a}_i &= \operatorname{concat}_{1 \leq h \leq \nHeads} \ \mathbf{a}_{i_h}
\label{eq:mha:concat}
\\
\modHidden'_i &= \mathbf{W}_{\mathrm{proj}} \mathbf{a}_i + \mathbf{b}_\mathrm{proj}
\label{eq:mha:output}
\end{align}

\includediagram{diagrams/gpt2/attention_multihead}{Structure of the multi-head attention mechanism. This example uses two positions $\mathbf{h}_1, \mathbf{h}_2$ and two attention heads. The actual model can in theory handle arbitrary sequence lengths and also has more attention heads.}



\subsubsection{Scaled dot-product attention}
\label{gpt2:attn_dot}
The scaled dot-product attention mechanism as outlined in \cref{diagrams/gpt2/attention_dot} and equations \eqref{eq:attn_dot_weight} to \eqref{eq:attn_dot_out} is the core part of multi-head attention described in \cref{gpt2:attn}.
Its job is to use the keys to find the positions most relevant to the query and amplify the values at these positions.
It is run in parallel for each of the $\nHeads$ attention heads. For simplicity, the head index $h$ is not repeated in the diagrams and equations here.
Note that scaled dot-product attention does not have any learnable parameters, the learnable parts like the creation of query, key and value vectors and projection after merging heads is done by multi-head attention.
The attention mechanism described here is more a less a more detailed explanation of the overview given in \cref{transformer:attn}:

\begin{enumerate}
\item The query vector $\mathbf{q}_i$ at position $i$ is compared with the key vector of each position $1 \leq i' \leq i$ using dot product to generate a score value $w_{i'i} \in \mathbb{R}$, as described in \cref{eq:attn_dot_weight}.
The scaling factor $\frac {1} {\sqrt{d_{head}}}$ helps improve gradient stability \citepage{4}{allyouneed}.

Positions after $i$ $(i' > i)$ are not considered because of the autoregressive nature of the model, i.e. the model is going to be used to predict the token at position $i+1$ based on the tokens up to position $i$. While the entire sequence is available during training, allowing for an efficient training process, during text generation only the tokens up to the current position $i$ will be available so this is what should be learned \citepage{5}{allyouneed}.

\item The scores $w_{i'i}$ are normalized in \cref{eq:attn_dot_softmax} using softmax so their sum is 1, this is required for the next step.

\item Using the scores from before as weights, the output vector $\mathbf{a}_i$ is "blended together" in \cref{eq:attn_dot_out} from each of the value vectors $\mathbf{v}_{i'}$ \cite{alammar-gpt2}:
\end{enumerate}

\begin{align}
	w_{i'i} &= \frac{1}{\sqrt{\dHead}} \ \mathbf{k}_{i'} \mathbf{q}_i
	\label{eq:attn_dot_weight}
	\\
	\operatorname{softmax}(w_{i'i}) &= \frac{\mathrm{e}^{w_{i'i}}} {\sum_{1 \leq i'' \leq i} \mathrm{e}^{w_{i''i}} }
	\label{eq:attn_dot_softmax}
	\\
	\mathbf{a}_i &= \sum _{1 \leq i' \leq i} \operatorname{softmax}(w_{i'i}) \mathbf{v}_{i'}
	\label{eq:attn_dot_out}
\end{align}

\includediagram{diagrams/gpt2/attention_dot}{Structure of the scaled dot-product attention mechanism. This diagram illustrates the computation of the attention output $\mathbf{a}_3$. The other output positions $\mathbf{a}_i$ are computed in a similar fashion, replacing $\mathbf{q}_3$ in the diagram with $\mathbf{q}_i$ and considering $\mathbf{k}_1 \ldots \mathbf{k}_i$ and $\mathbf{v}_1 \ldots \mathbf{v}_i$, giving the diagram $i$ columns.}


\subsubsection{Multi-layer perceptron}
\label{gpt2:mlp}


The multi-layer perceptron stage in the GPT2 block is a simple feed-forward neural network with one hidden layer \citepage{5}{allyouneed}, illustrated in \cref{diagrams/gpt2/mlp} and equations \eqref{eq:gpt2:gelu} to \eqref{eq:gpt2:mlp2}.

In contrast to the original transformer paper, GPT2 uses the Gaussian Error Linear Unit (GELU) activation function \citepage{5}{improvinglu}.
The GELU function for a real input $z$ is defined in \cref{eq:gpt2:gelu} as $z$ multiplied by the probability that a random variable $Z$ which follows a normal distrbution with mean 0 and variance 1 is less than $z$ \cite{gelu}.

The input vector $\modHidden_i$ is first subjected to a linear transformation with weights $\mathbf{W}_\textrm{MLP1}$ and biases $\mathbf{b}_\textrm{MLP1}$, followed by applying GELU to each component, to get the intermediate $\modHiddenStep{1}_i$, as described by \cref{eq:gpt2:mlp1}. The size of this intermediate vector is dependent on the dimensions of the learnable weight matrices and bias vectors ($\mathbf{W}_\textrm{MLP1}, \mathbf{b}_\textrm{MLP1}), (\mathbf{W}_\textrm{MLP2}, \mathbf{b}_\textrm{MLP2}$), which are typically configured such that the dimension of $\modHiddenStep{1}_i$ is $4 \cdot \dHidden$ \citepage{5}{allyouneed} \cite[source code]{HuggingFaceGPT2}.
Then another linear transformation happens in \cref{eq:gpt2:mlp2} with a different set of learnable parameters $\mathbf{W}_\textrm{MLP2}, \mathbf{b}_\textrm{MLP2}$ to get the output $\modHidden'_i$ with dimension $\dHidden$.

\begin{align}
	\mathrm{GELU}(z) &= z \cdot \mathbb{P}(Z<z) & z \in \mathbb{R}, Z \sim \mathcal{N}(0, 1)
	\label{eq:gpt2:gelu}
	\\
	\modHiddenStep{1}_i &= \mathrm{GELU}(\mathbf{W}_\textrm{MLP1} \modHidden_i + \mathbf{b}_\textrm{MLP1})
	\label{eq:gpt2:mlp1}
	\\
	\modHidden'_i &= \mathbf{W}_\textrm{MLP2} \modHiddenStep{1}_i + \mathbf{b}_\textrm{MLP2}
	\label{eq:gpt2:mlp2}
\end{align}

\includediagram{diagrams/gpt2/mlp}{Multi-layer perceptron stage of the GPT2 block. It is a simple feed-forward neural network with two linear steps with bias and an activation function in between.}

\section{Methods}
\label{methods}

Building on the results of \cite{teaching}, we wanted to replicate their good results for a small transformer model learning addition with up to 10 decimal digits \citepage{15}{teaching}.
We also wanted to focus on the harder task of integer multiplication with up to 10 decimal digits, which was studied in \cite{teaching} only for 2 decimal digit multiplicands.
Because we achieved no success in training this operation using our baseline setup (\cref{setup:baseline}), we experimented with a number of approaches that could help and are illustrated in this section: From simpler approaches like changing the distribution of samples in the training dataset (\cref{methods:sampling}), to more elaborate ones like automatically generated scratchpads (description of intermediate steps of a calculation) that augment the training samples (\cref{methods:scratchpad}).

\cite{teaching} also studied the (irrational) square root function, with the output being approximated by its first five decimal digits. We wanted to stick to integer operations and thus included the integer square root function (\cref{methods:ops:isqrt}) to our studied operations to see how hard it is to learn compared to addition and multiplication, and how it would respond to the techniques we tried for improving multiplication performance.

\subsection{Tokenization}
\label{methods:token}

We stick to a simple tokenization scheme where each character in the string is converted to one token ID based on its ASCII value, meaning our model supports token IDs from 0 to 127, thus $\dVocab=128$ for all our models. Token ID 0 is reserved to denote the end of the token sequence, it is appended to the ASCII token sequence.

\begin{lstlisting}
"1+2=3" -> [49, 43, 50, 61, 51, 0]
\end{lstlisting}


\subsection{Model choice}
\label{methods:model}

We use a transformer model with mostly the same GPT2 architecture described in \cref{gpt2}, with parameters $\dVocab=128, \nBlocks=6, \dHidden=384, \nHeads=6, \dHead=64$ (these parameters are explained in \cref{gpt2:parameters}).
This means we use the same model architecture and configuration parameters as in \cite{teaching} and thus get a model with about 11M parameters \citepage{3}{teaching}. This is considerably smaller than the versions of GPT2 released by OpenAI with about 100M to 1B parameters (\cref{gpt2:parameters}).
Since our resources in terms of hardware are limited and we train for single-purpose transformer models that can only do one specific arithmetic task, we consider this acceptable.
We do however briefly experiment with larger model sizes in some experiments described in \cref{setup:modelsize}.

The maximum sequence length of the model, $\nPos$, gets adjusted based on the string lengths of the training data for that particular experiment.


\subsection{Arithmetic operations}
\label{methods:ops}

We study the performance of transformer models on three different arithmetic operations: Addition and multiplication are standard arithmetic operations and common building blocks of more complex operations like exponentiation and evaluation of polynomial functions. Their learnability by transformer models has been investigated in \cite{teaching} and \cite{visual} among others. Integer square root, on the other hand, has to our knowledge not been studied before in the context of transformer networks.

\subsubsection{Addition}
\label{methods:ops:add}

Let $a, b$ be positive integers and let $c=a+b$. The model is then trained on strings of the form $a+b=c$.


\subsubsection{Multiplication}
\label{methods:ops:mul}

Let $a, b$ be positive integers and let $c=a \cdot b$. The model is then trained on sequences of the form $a \cdot b=c$.

\subsubsection{Integer square root}
\label{methods:ops:isqrt}

In addition to addition and multiplication, we also study the learnability of the integer square root function. Instead of the regular square root function with irrational values, we use integer square root. This is consistent with the addition and multiplication operations in that it is an integer to integer mapping, with exactly one correct, finite answer for each input number, yet being equivalent to approximating the regular square root function.

The integer square root function $\lfloor \sqrt{x} \rfloor$ can be defined as $\max(\{ k \in \mathbb{N} | k^2 \leq x \})$.

A model that can compute the integer square root for integers with a certain number of decimal digits can also approximate the regular square root function:

\begin{itemize}
	\item $\sqrt{\frac{1}{2}} \approx 0.7071067811$.
	\item $\sqrt{\frac{1}{2}} = \sqrt{0.5} = 10^{-5} \sqrt{10^{10} \cdot 0.5}
	= 10^{-5} \sqrt{5 \cdot 10^9}$.
	\item $\lfloor \sqrt{5 \cdot 10^9} \rfloor = 70710$.
	\item $\sqrt{\frac{1}{2}} \approx 0.70710$.
\end{itemize}

We train the model of sequences of the form $x: k$ where $k$ is the integer square root of $x$.


\subsection{Sampling methods}
\label{methods:sampling}

The distribution based on which training samples are picked can have a big impact on the performance of machine learning models. To find out how much of an improvement can be had on the operations we study just by changing sampling methods, we experimented with multiple sampling methods: Basic sampling, from-zero sampling, uniform digits sampling and uniform bits sampling.

\subsubsection{Basic sampling}
\label{methods:sampling:basic}

For the addition and multiplication tasks with $n$ digits, the operands $a$ and $b$ are independently sampled uniformly from the range $[{10}^{n-1}, 10^n - 1]$, which ensures both operands have exactly $n$ decimal digits. We train the models for $n=3$, $n=5$,  and $n=10$ operand digits, giving a sample space of approximate size $10^6$, $10^{10}$, $10^{20}$ respectively.

For the square root operation with $n$ operand digits, we also sample the operand $x$ uniformly from the range $[{10}^{n-1}, 10^n - 1]$. However, because square root is a unary operator, we double the number of decimal digits of the single operand to make sure we have a similarly large space of possible samples. Therefore we train the models for $n=6$, $n=10$,  and $n=20$ operand digits, giving a sample space of approximate size $10^6$, $10^{10}$, $10^{20}$ respectively.

\subsubsection{From-zero sampling}
\label{methods:sampling:fromzero}

For basic sampling the limits of operand sampling were set as $[10^{n-1}, 10^n - 1]$ to include all $n$-digit numbers, where n is the number of operand digits. From-zero sampling instead uses the limits $[0, 10^n - 1]$, to also include operand numbers with fewer decimal digits.

\subsubsection{Uniform digits sampling}
\label{methods:sampling:digits}

For this sampling method, first a number of decimal digits $k$ is picked uniformly at random from $[1, n]$. Then the operands are picked uniformly at random from $[10^{k-1}, 10^k - 1]$, to ensure they are $k$-decimal digit integers.

\subsubsection{Uniform bits sampling}
\label{methods:sampling:bits}

Let $p$ be the smallest integer so that $2^p \geq {10}^n$. This ensures all numbers with $n$ decimal digits are covered by the range $[1, 2^p]$.
A number of bits $k$ is picked uniformly at random from $[1, p]$. Then the operands are picked uniformly at random from $[2^{k-1}, 2^k - 1]$ to ensure they are $k$-bit integers.




\subsection{Result reversal}
\label{methods:reversal}

Providing some intermediate steps how to arrive at the result within each training sample as opposed to just input/output pairs can make a substantial improvement to final accuracy when transformer models learn addition, especially with more operand digits (\citepage{15}{teaching}).
Similar to \cite{teaching}, we tested two different strategies: Reversing the digits of the result, and detailing a step-by step arithmetic process to compute the result, similar to doing addition or multiplication on paper or in an algorithm. This we refer to as a "scratchpad". We expanded the experiments in \cite{teaching} beyond addition to multiplication and integer square root with varying operand sizes, testing whether result reversal or scratchpad could also help with learnability for these harder operations.

When computing a sum or product by hand, one starts from the right, computing the least significant digits of the result before moving left to the more significant digits. However, in the baseline experiments, the model needs to predict the most significant digit first and the least significant last.

To accurately predict the most significant digit of a sum, one generally needs to have the exact result of the sum computed already (think about operands where the sum is close to wrapping on the most-significant digit, i.e., small changes in the operand can cause a change in the most significant digit of the sum).

In contrast, knowing the least significant digits of the operands is sufficient to compute the least significant digit of the sum. Thus, it seems reasonable to assume that computing a sum "backwards" would be easier for a model than computing it in normal digit order.


To allow the model to first write down the reversed sum before "committing" to the actual result, we modify the training samples: After the = sign, which is the point from which the model is asked to predict the rest of the training sample (which is the sum that we want to know to test addition capability), we put the reversed digits of the sum in brackets before the normal sum.
For example, the training sample \verb!456+789=1245!
would be transformed into:

\begin{lstlisting}[
	float,
	caption={Result reversal training sample for the addition task 456+789=1245.},
	label={lst:reversal}
	]
	456+789=[5421]1245
\end{lstlisting}

For the multiplication and square root tasks, the digits of the product or integer square root are reversed in the same fashion - first writing the reversed result in brackets before writing the actual result.

\noindent
Due to the augmented training samples, we also change the way we evaluate correctness when evaluating model performance: For the baseline experiments, we compared the whole output after the = sign. For the result reversal experiments, we first strip the part in brackets (the reversed result) and then only compare what comes after it (the actual result).




\FloatBarrier
\subsection{Scratchpads}
\label{methods:scratchpad}


Compared to augmenting the training sample with the reversed result, which is a simple process that does not tell the model in detail how to arrive at the result, a scratchpad is a detailed, step-by-step, digit-by-digit walkthrough of a basic algorithm solving the task, decomposing it into small intermediate chunks.

Similar to result reversal, we put the scratchpad (the intermediate steps) in brackets before the correct result. The model then, when prompted, outputs this scratchpad for the task at hand, thus executing the algorithm, then in the ideal case arrives at the correct output after the brackets.

\subsubsection{Scratchpad: Addition}
\label{add_scratchpad}

For the addition task, the scratchpad illustrates basic digit-by digit addition of two integers: Add two digits of operands plus carry from the previous step, separate ones and tens, take ones as a digit of the sum, keep tens as carry. Move one digit to the left and repeat until the operands have been fully processed. For the training sample \verb!456+789=1245! the sample with scratchpad looks like \cref{lst:scratchpad_add}.

\begin{lstlisting}[
	float,
	caption={Example scratchpad for the addition task 456+789=1245.},
	label={lst:scratchpad_add}
	]
	456+789=[69051|58141|47121]1245
\end{lstlisting}

Each step is separated by \verb!|! characters. The two summand digits, previous carry, sum digit, and new carry are written for each step.
The scratchpad is deliberately kept minimal because of the increase in training time with increased sample length.

\label{training_time_growth}
This increase in training time is not due to an increase in model parameters, but rather due to the model's attention mechanism comparing one position to all previous positions (\cref{gpt2:attn}). This causes training time to increase quadratically with sequence length. Memory requirements also increase quadratically due to the resulting attention matrix when comparing each position with each previous one.

\FloatBarrier
\subsubsection{Scratchpad: Multiplication}
\label{mul_scratchpad}

Basic multiplication works by doing single-digit multiplication between the multiplicand and all digits of the multiplier, then adding an appropriate amount of zero digits to each intermediate product, then computing the sum of all the intermediate products. This process forms the basis for our multiplication scratchpad. For example, the multiplication task \verb!67*89=5963!
is transformed into the scratchpad-augmented form seen in \cref{lst:scratchpad_mul}.

\begin{lstlisting}[
	float,
	caption={Example scratchpad for the multiplication task 67*89=5963.},
	label={lst:scratchpad_mul}
	]
	67*89=[
	*80{78065,68535}5360,
	*9{79036,69606}603
	|
	{03,0,3,0;60,0,6,0;36,0,9,0;5,0,5,0}
	]5963
\end{lstlisting}

\noindent
The multiplication scratchpad is divided by \verb!|! into the part computing the digit products and the part doing the final addition.
Each digit product has its own scratchpad enclosed in \verb!{}!.

The computation of the digit products happens like this (example for the multiplication 67*89):

\begin{lstlisting}
	*80{78065,68535}5360
\end{lstlisting}

First comes the multiplier (80 in this case), then the multiplication steps in curly braces, then the result of the digit multiplication.
For each multiplication step, we list five digits: First, the corresponding digit of the multiplicand (first 7, then 6 for 67), then the digit of the multiplier (always 8 for 80), then the previous carry, then the output digit, then the current carry.

The final addition happens like this ($5360+603$ for our example):

\begin{lstlisting}
	{03,0,3,0;60,0,6,0;36,0,9,0;5,0,5,0}
\end{lstlisting}

Each addition step is separated by semicolons. The parts of each addition step are separated by commas. For each addition step, we first list the corresponding digits of the summands (e.g., 0 and 3 for the first step of $5360+603$). Then we list the previous carry, then the output digit, and finally, the current carry.

\FloatBarrier
\subsubsection{Scratchpad: Square root}
\label{sqrt_scratchpad}

A straightforward way to compute the integer square root of an integer $n$ is by using a binary search algorithm is illustrated by \cref{lst:binarysearch}.
The integer square root scratchpad we use follows the steps of this algorithm, illustrated on the training sample \verb|123:11| in \cref{lst:binarysearchexample}.

The steps of the binary search algorithm are in brackets, separated by pipes.
Consider the first step for example:
\begin{lstlisting}
	{0,123} 61*61=3721
\end{lstlisting}

The current search range is in braces, for this step it is from 0 to 123.
Then, the midpoint of the search range (61) is squared to 3721.
We are looking for the integer square root of 123, which is the largest integer whose square is at most 123 (\cref{methods:ops:isqrt}). Because 3721 is larger than $n=123$, the search range in the next step gets restricted to the lower half, which is from 0 to 61.
The algorithm ends when the search range consists of at most two integers (11 and 12 in this case), at which point the larger one is tested, as $12*12=144$ is too large, $11$ is the correct integer square root.

It would have been nice to include the multiplication scratchpad from \cref{mul_scratchpad} for the square operations of this algorithm, however to keep training feasible we have to keep the samples below a few thousand characters. Considering the fact that we want to compute square roots of 6, 10, and 20 digit integers as in the baseline experiments, this is not possible.
Thus we simply include the square operations without their own scratchpads and hope this illustration of binary search still provides improved accuracy compared to the baseline.


\begin{lstlisting}[
	float,
	caption={Binary search algorithm for computing the integer square root.},
	label={lst:binarysearch}
]
low, high = 0, n
while high - low > 1:
  mid = floor((low + high) / 2)
  if mid*mid <= n:
    low, high = mid, high
  else:
    low, high = low, mid 
if high * high <= n:
  return high
else:
  return low
\end{lstlisting}



\begin{lstlisting}[
	float,
	caption={Scratchpad example for computing the integer square root of 123.},
	label={lst:binarysearchexample}
	]
123:[
{0,123} 61*61=3721| 
{0,61} 30*30=900|
{0,30} 15*15=225|
{0,15} 7*7=49|
{7,15} 11*11=121|
{11,15} 13*13=169|
{11,13} 12*12=144|
{11,12}
]11
\end{lstlisting}





\FloatBarrier
\subsection{Attention heatmaps}
\label{methods:heatmap}

To get some degree of insight into how the trained model computes the results for the arithmetic tasks presented to it, we generated 2D heatmaps showing activation of the attention mechanism for various input sequences. Our process for creating heatmap diagrams roughly follows the attention heatmaps presented in \cite{bertsecrets} and \cite{analyzingheads}:

The model consists of a pre-processing stage, a number of GPT2 blocks and finally a post-processing stage (\cref{diagrams/gpt2/overview}). Inside the GPT2 blocks, the core part is the attention mechanism (\cref{gpt2:attn_dot}): The input vector at each position is transformed into a query, a key and a value vector. The query vector at one position is then compared with the key vectors at every position, to generate an attention score from 0 to 1 at each position, which is then used to blend the value vectors together to get the output at this position (\cref{diagrams/gpt2/attention_dot}).
Before running the attention mechanism, the input vector is split into multiple attention heads, on each of which a separate attention mechanism is run, whose outputs are then combined (\cref{diagrams/gpt2/attention_multihead}).

We can thus generate, for a given token sequence, a diagram that, for each GPT2 block and each attention head inside that block, contains a 2D heatmap that shows which token / position attends to which other position.
We create two types of heatmap diagrams: First we started with full diagrams showing each block and attention head, then for clarity we simplified the other diagrams to averaged diagrams with just a single 2D image:

\begin{itemize}
	\item \textbf{Full diagram:} For each GPT2 block of the model and each attention head per block, generate a 2D image with color values corresponding to how strongly attention output for one position is influenced by inputs at each position (\cref{gpt2:attn_dot}).
	\item \textbf{Averaged diagram:} When looking at the full diagrams, we noticed meaningful attention patterns only in the first GPT2 block, with no clear difference between attention heads. Also, full diagrams can get quite large, especially when showing the model's handling of longer sequences (result reversal or scratchpad). We thus reduced diagrams to only showing the first layer and averaging among all attention heads. This is in contrast to \cite{bertsecrets} which focused more on the different roles of individual attention heads. For out purposes of showing some meaningful patters in the model's attention mechanism, averaging worked fine however and presented a single 2D image.
\end{itemize}

Examples of our full and averaged heatmap diagrams can be seen in \cref{results:heatmap}.


\section{Experimental setup}


\subsection{General setup}
\label{setup:general}


\subsubsection{Libraries / Frameworks}

We use the HuggingFace transformers library with PyTorch backend which takes care of implementing transformer networks and the training process. Based on this, we wrote some Python code that implements generating training samples, running model training, evaluating model performance, logging results and generating plots. The generation of plots from the result logs utilized the matplotlib Python library.

\subsubsection{Hardware}
\label{expsetup:hardware}

The experiments were run on Linux machines with NVIDIA GPUs that were rented and accessed via a GPU renting website. We ensured each system had a reasonably recent NVIDIA GPU with at least 16GB VRAM to ensure our experiment computations fit inside GPU memory.

\subsubsection{Model implementation}

We use the GPT2LMHeadModel class of the HuggingFace transformers library which implements a GPT2-style decoder-only transformer model.
We adjust the model configuration so it matches nanoGPT, a basic lightweight GPT2 implementation which was also used for the experiments in REF. 
The model uses a vector size of 384, 6 attention heads and 6 transformer blocks.

\subsubsection{Data format}

For addition and multiplication, the operands $a$ and $b$ are encoded into the string based on their decimal digit representation, the same applies to the sum / product $c$. 
To avoid uneven string/sequence length, $c$ is encoded with leading zeroes up to the maximum possible sum / product length based on the operand size. For example, for the addition task with 3 digits, the sum is padded to 4 digits:

\begin{lstlisting}
	"123+456=0579"
\end{lstlisting}

For the square root operation, we encode a sample like this:

\begin{lstlisting}
	"500000:707"
\end{lstlisting}

For even $n$, the integer square root of a $n$-digit decimal number always has $\frac{n}{2}$ decimal digits, thus no padding is needed.

\subsubsection{Datasets}

Based on the sampling and formatting methods described above, we create 3 separated datasets for the given arithmetic operation, operand size and dataset size:

\begin{itemize}
	\item \textbf{Training dataset:} This dataset is used to optimize the model weights during training. Its size varies from 1k to 100k samples as part of the experiment.
	\item \textbf{Validation dataset:} A separate dataset with 1k samples. Loss on this dataset is used to decide when to stop training (\cref{early_stopping}). Accuracy on this dataset is also tracked.
	For some experiments (\cref{expsetup:scratchpad}) we had to lower the size of the validation dataset to avoid accuracy tracking slowing down training too much.
	\item \textbf{Test dataset:} Another separate dataset with 1k samples. This dataset is used at the end of each training run to compute the final accuracy numbers.
\end{itemize} 

\subsubsection{Training}

Training is done using the HuggingFace Trainer class with some customization:

\subsubsection{Weight initialization}

The model weights are initialized using the default distributions as defined in the GPT2LMHeadModel class of the HuggingFace transformers framework.

\subsubsection{Optimizer}

Training uses the AdamW optimizer with default parameters as defined by the HuggingFace transformers Trainer / TrainingArguments class.

\subsubsection{Batch size}
\label{expsetup:batchsize}

For the baseline experiments, we used a fixed batch size of 256. We picked this batch size as it was also used in \cite{teaching} and it fits the memory of our training GPU.

\subsubsection{Learning rate scheduling}
\label{expsetup:learnrate}

The learning rate is adjusted using a linear schedule with warmup steps, with default parameters as defined by the HuggingFace transformers framework.
For computing the learning rate, we assume a maximum of 200 epochs for training.

\subsubsection{Early stopping}
\label{early_stopping}

We use an early stopping strategy where training is aborted if for 5 epochs there is no new minimum in loss on the validation dataset (minimum improvement $\epsilon=10^{-4}$).

\subsubsection{Generation}

To generate answers to a prompt string like "1+2=", we use auto-regressive generation as is standard practice with transformer decoder models (REF). We use greedy decoding, always selecting the token with the highest probability. Generation stops when token ID 0 (stop token) is generated or the model max sequence length is reached.
\label{model_generation}

\subsubsection{Evaluation}

During the training process, multiple metrics are tracked:

\begin{itemize}
	\item \textbf{Average batch loss:}
	For each training sample, the model outputs a token probability distribution at each position of the sample. This is compared with the correct next token at this position to compute a cross-entropy loss number. This loss is averaged among all positions of a sample and all samples in the current batch. To reduce statistical noise, the HuggingFace trainer class also averages these batch loss numbers among multiple consecutive training steps.
	
	\item \textbf{Validation loss:}
	At the end of each epoch, the average loss over all samples in the validation dataset is computed. This number is used to track training progress and stop training if no more progress is made.
	
	\item \textbf{Validation accuracy:}
	For each sample in the validation dataset, the string is split at the "=" character, yielding a prompt and answer string (e.g. prompt "1+2=" and corresponding answer "3"). For the prompt string, the model's answer is generated as described in \cref{model_generation}. This answer is then compared to the correct answer. The ratio of samples where the model's answer matches the correct answer gives the accuracy.
	
	\item \textbf{Test accuracy:}
	It is evaluated just like validation dataset accuracy, but on the test dataset. This gives the final accuracy numbers reported in the tables below.
\end{itemize}


\subsubsection{Error analysis}
\label{setup:error}

We wanted to analyze not just the performance of the final models in terms of the percentage of correctly solved tasks, but also take a look at the types of errors the model makes.
For this, we picked the first few tasks from the test dataset that the model answered incorrectly. If there were no errors in the test dataset, we created an additional error analysis dataset with 100k samples to look for rare errors.
Error


\subsection{Baseline experiments}
\label{setup:baseline}

To test the ability of transformer networks to learn arithmetic operations, and to see whether we can replicate the results of \cite{teaching}, we first run some baseline experiments. The general setup is as described in \cref{setup:general} and the training samples for the studied operations were created as described in \cref{methods:ops}.

We run a total of 27 baseline experiments, comparing 3 operations, 3 operand sizes per operation, and 3 training dataset sizes:
 
\begin{itemize}
	\item \textbf{Addition:}
	\begin{itemize}
		\item 3, 5, and 10 digit operands
		\item 1k, 10k, and 100k training samples
	\end{itemize}
	\item \textbf{Multiplication:}
	\begin{itemize}
		\item 3, 5, and 10 digit operands
		\item 1k, 10k, and 100k training samples
	\end{itemize}
	\item \textbf{Integer square root:}
	\begin{itemize}
		\item 6, 10, and 20 digit operands
		\item 1k, 10k, and 100k training samples
	\end{itemize}
\end{itemize}








\subsection{Model size experiments}
\label{setup:modelsize}

To see whether changing the number of parameters of the model improves accuracy, we tried each of the following changes to model dimensions:

\begin{itemize}
	\item Change the hidden vector size $\dHidden$ from 384 to either 192 or 768. This causes a substantial change to the number of learnable model parameters: The model with $\dHidden=192$ has about 2.7M parameters, the baseline model with $\dHidden=384$ about 11M parameters, and the model with $\dHidden=768$ about 43M parameters. Thus doubling $\dHidden$ roughly quadruples the number of parameters (due to matrices for transformations on vectors twice the size having 4 times as many parameters).
	\item Change the number of GPT2 blocks $\nBlocks$ from 6 to either 3 or 12. This also changes the number of parameters, but in a linear fashion: The model with 3 blocks has 5.4M parameters, and the model with 12 blocks has 21M parameters. 
	\item Change the number of attention heads $\nHeads$ from 6 to either 3 or 12. This has no effect on the number of model parameters, it only affects the "partitioning" of the parameters inside the model.
\end{itemize}

We did not try to combine these changes, each experiment only had one parameter of the model architecture changed.
Due to time/hardware constraints, we limited experiments to the following tasks:

\begin{itemize}
	\item Addition with 10 operand digits, 100k training dataset.
	\item Multiplication with 5 operand digits, 100k training dataset.
\end{itemize}

We picked these two tasks from the baseline experiments because addition with 10 digits could be learned well with the 100k dataset, while still being harder than 3 or 5 digit addition, allowing more chances to see a difference in model performance in terms of how many optimization steps it takes to converge.
Similarly, multiplication with 5 digits and 100k training dataset was chosen because the baseline model could not learn it and achieved zero accuracy. At the same time, learning 5 digit multiplication seemed somewhat realistic with a more capable model architecture, considering that 3 digit multiplication could be learned at a low but not zero (22\%) accuracy.

In summary, we run these 12 experiments with 6 different model changes attempted on two operations and fixed operand digits and dataset size per operation:

\begin{itemize}
	\item \textbf{Addition:} 10 operand digits, 100k training dataset, baseline model but with one of these changes: $\dHidden=192, \dHidden=768, \nBlocks=3, \nBlocks=12, \nHeads=3$ or $\nHeads=12$.
	\item \textbf{Multiplication:} 5 operand digits, 100k training dataset, baseline model but with one of these changes: $\dHidden=192, \dHidden=768, \nBlocks=3, \nBlocks=12, \nHeads=3$ or $\nHeads=12$.
\end{itemize}


\subsection{Sampling strategy experiments}
\label{setup:sampling}

The timing needed.

\subsection{Result reversal experiments}
\label{setup:reversal}



Augmenting the training samples with the reversed result of the arithmetic operation results in increased sample length (\cref{tbl:reverse_sample_lengths}).
This means we have to increase the number of positions in the model configuration to handle these longer sequences. This does lead to more model parameters due to each position having a learnable positional embedding vector (\cref{gpt2:preproc}). The increase however is just a minute fraction of the total parameters (e.g., 10.710 million parameters for the 10-digit addition baseline requiring 34 positions, and 10.715 million parameters for 10-digit addition with result reversal requiring 47 positions). This minute increase does not necessitate other changes to the training regime such as reduced batch sizes or fewer epochs - we can keep the default batch size of 256 and epoch limit of 200 (\cref{expsetup:batchsize}, \cref{expsetup:learnrate}).


\includeTable
{
	task; lenNormal; lenReverse\\
	Addition (3 digits); 13; 19 \\
	Addition (5 digits); 19; 27 \\
	Addition (10 digits); 34; 47 \\
	Multiplication (3 digits); 15; 23 \\
	Multiplication (5 digits); 23; 35 \\
	Multiplication (10 digits); 43; 65 \\
	Square root (6 digits); 11; 16 \\
	Square root (10 digits); 17; 24 \\
	Square root (20 digits); 32; 44 \\
}
{tbl:reverse_sample_lengths}
{
	Sequence lengths for training samples (prompt string + response string + end token) for various tasks and operand sizes.
}
{%
	columns/task/.style={column name={Task}},
	columns/lenNormal/.style={column name={\begin{tabular}{c} Sequence length \\ (baseline) \\ \end{tabular}}},
	columns/lenReverse/.style={column name={\begin{tabular}{c} Sequence length \\ (result reversal) \\ \end{tabular}}}
}

\subsection{Scratchpad experiments}
\label{setup:scratchpad}



Augmenting the training samples with a scratchpad that details the steps of the arithmetic operation to be performed results in a large increase in sequence length (\cref{tbl:scratchpad_sample_lengths}). This increase is especially noticeable for the multiplication operation (which is composed of several single-digit multiplications followed by a large addition) and the square root operation (which is done by binary search with squaring at every step).

These large sequences of sometimes well beyond 1000 characters necessitate some changes to the training regime. Based on the available hardware, we need to make training batches and the intermediate results during forward/backward pass fit into GPU memory (\cref{expsetup:hardware}). The smaller batches in turn mean more optimization steps per epoch, thus taking more time to do training for a given number of epochs. At the same time, a training batch with longer sequences results in larger matrices during optimization, thus taking more time per batch than the same batch size with shorter sequences.
Time for evaluating the model (generating answers for the samples in the test/validation dataset to test accuracy) also rises proportionally to the lengths of the sequences being generated, meaning it can become a significant part of training time when sequences are long due to an elaborate scratchpad being trained on.

We want to keep the duration for a single experiment (e.g., square root, using scratchpad, 20 digits, 100k dataset) below 24 hours on the available hardware (\cref{expsetup:hardware}).
We still want to run all experiments, even those with the largest dataset size of 100k, thus we made the compromise of reducing the maximum number of training epochs for the experiments that would take too much time otherwise. This does have the downside of the models being undertrained (unless early stopping would have halted training before the epoch limit) but it seems like the best choice given the time and hardware constraints.



For the addition operation (\cref{tbl:scratchpad_training_add}) the only change that needed to be made was reducing the maximum epochs from 200 to 20 for experiments with the large 100k training dataset.

For the multiplication operation (\cref{tbl:scratchpad_training_mul}) the batch sizes needed to be reduced depending on the sequence length. Also the number of epochs needed to be reduced for the experiments with 10k and 100k datasets.

For the square root operation (\cref{tbl:scratchpad_training_sqrt}), due to time and memory constraints (see \cref{training_time_growth}), the batch sizes and epoch counts needed to be cut even more, even down to a batch size of 1 and epoch count of 1 for the task with 20 operand digits and 100k dataset. Even though this results in much less training than the comparable baseline task, we still think it is better trying to at least get a glimpse of training results for this task instead of not including it at all.

In addition to the experiment-specific changes mentioned above, we also reduced the size of the validation dataset from 1000 to 10 elements and perform accuracy evaluations not once every 100 steps during logging but only at the end of each epoch. This gives a less detailed report of accuracy development during training but saves significant evaluation time. Accuracy on the test dataset, which is unchanged at 1000 elements, is performed only at the end of the experiment, same as for the baseline, thus final accuracy numbers in the results table do not suffer in precision.
For testing accuracy, we only consider the final result after the scratchpad in brackets, i.e. the correctness of the intermediate steps is not verified.



\includeTable
{
	task; lenNormal; lenScratch\\
	Addition (3 digits); 13; 32 \\
	Addition (5 digits); 19; 50 \\
	Addition (10 digits); 34; 95 \\
	Multiplication (3 digits); 15; 159 \\
	Multiplication (5 digits); 23; 356 \\
	Multiplication (10 digits); 43; 1181 \\
	Square root (6 digits); 11; 602 \\
	Square root (10 digits); 17; 1447 \\
	Square root (20 digits); 32; 5083 \\
}
{tbl:scratchpad_sample_lengths}
{
	Sequence lengths for training samples (prompt string + response string + end token) for various tasks and operand sizes.
}
{%
	columns/task/.style={column name={Task}},
	columns/lenNormal/.style={column name={\begin{tabular}{c} Sequence length \\ (baseline) \\ \end{tabular}}},
	columns/lenScratch/.style={column name={\begin{tabular}{c} Sequence length \\ (scratchpad) \\ \end{tabular}}}
}



\includeTable
{
	digits; batch; batchScratch; epochs; epochsScratch\\
	3 digits,  1k;  256; 256; 200; 200\\
	3 digits,  10k;  256; 256; 200; 200\\
	3 digits,  100k;  256; 256; 200; 20\\
	5 digits,  1k;  256; 256; 200; 200\\
	5 digits,  10k;  256; 256; 200; 200\\
	5 digits,  100k;  256; 256; 200; 20\\
	10 digits,  1k;  256; 256; 200; 200\\
	10 digits,  10k;  256; 256; 200; 200\\
	10 digits,  100k;  256; 256; 200; 20\\
}
{tbl:scratchpad_training_add}
{
	Training regime changes for the \textbf{addition} operation.
}
{%
	columns/digits/.style={column name={\begin{tabular}{c}
				Operand digits, \\
				dataset size
	\end{tabular}}},
	columns/dataset/.style={column name={\begin{tabular}{c}
				Dataset \\
				size
	\end{tabular}}},
	columns/batch/.style={column name={\begin{tabular}{c}
				Batch size \\
				(baseline)
	\end{tabular}}},
	columns/batchScratch/.style={column name={\begin{tabular}{c}
				Batch size \\
				(scratchpad)
	\end{tabular}}},
	columns/epochs/.style={column name={\begin{tabular}{c}
				Max. \\
				\# epochs \\
				(baseline)
	\end{tabular}}},
	columns/epochsScratch/.style={column name={\begin{tabular}{c}
				Max. \\
				\# epochs \\
				(scratchpad)
	\end{tabular}}}
}

\includeTable
{
	digits; batch; batchScratch; epochs; epochsScratch\\
	3 digits,  1k;  256; 128; 200; 200\\
	3 digits,  10k;  256; 128; 200; 50\\
	3 digits,  100k;  256; 128; 200; 5\\
	5 digits,  1k;  256; 64; 200; 200\\
	5 digits,  10k;  256; 64; 200; 50\\
	5 digits,  100k;  256; 64; 200; 5\\
	10 digits,  1k;  256; 16; 200; 200\\
	10 digits,  10k;  256; 16; 200; 50\\
	10 digits,  100k;  256; 16; 200; 5\\
}
{tbl:scratchpad_training_mul}
{
	Training regime changes for the \textbf{multiplication} operation.
}
{%
	columns/digits/.style={column name={\begin{tabular}{c}
				Operand digits, \\
				dataset size
	\end{tabular}}},
	columns/dataset/.style={column name={\begin{tabular}{c}
				Dataset \\
				size
	\end{tabular}}},
	columns/batch/.style={column name={\begin{tabular}{c}
				Batch size \\
				(baseline)
	\end{tabular}}},
	columns/batchScratch/.style={column name={\begin{tabular}{c}
				Batch size \\
				(scratchpad)
	\end{tabular}}},
	columns/epochs/.style={column name={\begin{tabular}{c}
				Max. \\
				\# epochs \\
				(baseline)
	\end{tabular}}},
	columns/epochsScratch/.style={column name={\begin{tabular}{c}
				Max. \\
				\# epochs \\
				(scratchpad)
	\end{tabular}}}
}

\includeTable
{
	digits; batch; batchScratch; epochs; epochsScratch\\
	6 digits,  1k;  256; 32; 200; 200\\
	6 digits,  10k;  256; 32; 200; 50\\
	6 digits,  100k;  256; 32; 200; 5\\
	10 digits,  1k;  256; 8; 200; 200\\
	10 digits,  10k;  256; 8; 200; 50\\
	10 digits,  100k;  256; 8; 200; 5\\
	20 digits,  1k;  256; 1; 200; 100\\
	20 digits,  10k;  256; 1; 200; 10\\
	20 digits,  100k;  256; 1; 200; 1\\
}
{tbl:scratchpad_training_sqrt}
{
	Training regime changes for the \textbf{square root} operation.
}
{%
	columns/digits/.style={column name={\begin{tabular}{c}
				Operand digits, \\
				dataset size
	\end{tabular}}},
	columns/dataset/.style={column name={\begin{tabular}{c}
				Dataset \\
				size
	\end{tabular}}},
	columns/batch/.style={column name={\begin{tabular}{c}
				Batch size \\
				(baseline)
	\end{tabular}}},
	columns/batchScratch/.style={column name={\begin{tabular}{c}
				Batch size \\
				(scratchpad)
	\end{tabular}}},
	columns/epochs/.style={column name={\begin{tabular}{c}
				Max. \\
				\# epochs \\
				(baseline)
	\end{tabular}}},
	columns/epochsScratch/.style={column name={\begin{tabular}{c}
				Max. \\
				\# epochs \\
				(scratchpad)
	\end{tabular}}}
}





\subsection{Attention heatmaps}
\label{setup:heatmap}

We set up the following attention heatmap experiments: ...

%\section{Experimental results}


\subsection{Addition baseline}

For the addition operation, we can see in \cref{tbl:baseline_add} that addition up to 10 digits can be learned, as long as the training dataset has at least 10k samples. Accuracy numbers are high, but short of 100\% for 5 and 10 digit addition, meaning there are quite a few errors left in the validation dataset. Only 3 digit addition achieved perfect accuracy.

Considering the development of accuracy during training, we can see in \cref{fig:baseline_add} that accuracy flatlines until a few hundred to a few thousand optimization steps are performed. More operand digits mean more optimization steps to reach the point of accuracy leaving zero. Interestingly, larger dataset sizes also require slightly more optimization steps to leave zero accuracy.
After leaving zero, accuracy quickly jumps to about 80\%, then slowly keeps climbing to close to 100\%.
Based on these accuracy curves, it does not seem unreasonable that training for longer that early stopping permits (\cref{early_stopping}) would have resulted in even better accuracy. Investigating the type and parameters of our early stopping criterion could be worthwhile.

\includeAccuracyTable{experiment_results/baseline_final_accuracy/add.csv}{tbl:baseline_add}{Final model accuracy on the test dataset for the addition operation for the given training dataset sizes and operand digits.}{}

\includePDFPlot{experiment_results/baseline_accuracy/add.pdf}{fig:baseline_add}{Development of accuracy on the validation dataset when training for the addition task for different operand digits and dataset sizes. Square markers represent the point training was terminated by early stopping.}

\subsection{Multiplication baseline}

For the multiplication task, we can see, in \cref{tbl:baseline_mul}, a very different picture to the addition task. Accuracy never gets above zero, except for the 3 digits, 10k dataset experiment, where it reaches about 1\%, and the 3 digits, 100k dataset experiment, where it reaches about 22\%.
This suggests that learning multiplication, all else being equal, is harder than learning addition.

When looking at the accuracy curves in  \cref{fig:baseline_mul}, we can see that for the only experiment where accuracy clearly leaves zero (3 digit multiplication, 100k dataset) accuracy does not jump quickly to a certain value as it did with addition, but slowly keeps climbing until early stopping terminates training.
For the experiments that do not achieve any accuracy, we can see that the number of optimization steps performed until training is aborted depends mostly on training dataset size and not much on operand digits, with the experiments with 1k dataset getting aborted after about 200 steps, and the experiments with 100k dataset getting aborted after about 40000 steps.

\includeAccuracyTable{experiment_results/baseline_final_accuracy/mul.csv}{tbl:baseline_mul}{Final model accuracy on the test dataset for the multiplication operation for the given training dataset sizes and operand digits.}{}

\includePDFPlot{experiment_results/baseline_accuracy/mul.pdf}{fig:baseline_mul}{Development of accuracy on the validation dataset when training for the multiplication task for different operand digits and dataset sizes. Square markers represent the point training was terminated by early stopping.}

\subsection{Square root baseline}

For the square root task, we can see, in \cref{tbl:baseline_sqrt}, accuracy numbers that lie somewhere between those for addition and multiplication.
For 6 digit integer square root, there was high but not perfect accuracy when using a 10k or 100k training dataset.
For 10 digit integer square root, accuracy was 63\% for the 100k dataset but remained low otherwise. For 20 digit integer square root, no samples in the validation dataset where predicted correctly.

When looking at the accuracy curves in \cref{fig:baseline_sqrt}, we can see curves that are more similar to those of the multiplication experiment than the addition experiments, with accuracy slowly climbing from zero. Once again, it looks like early stopping might prevent the model from achieving their full accuracy potential.

\includeAccuracyTable{experiment_results/baseline_final_accuracy/sqrt.csv}{tbl:baseline_sqrt}{Final model accuracy on the test dataset for the square root operation for the given training dataset sizes and operand digits.}{}

\includePDFPlot{experiment_results/baseline_accuracy/sqrt.pdf}{fig:baseline_sqrt}{Development of accuracy on the validation dataset when training for the square root task for different operand digits and dataset sizes. Square markers represent the point training was terminated by early stopping.}

\subsection{Error analysis}
\label{error_analysis}

To get a better understanding of the presented accuracy numbers, we took a closer look at samples where the answer of the model differed from the correct one. This was done for tasks where the final model showed perfect or near-perfect accuracy:

\begin{itemize}
    \item Addition (5 digits, 100k training samples)
    \item Addition (10 digits, 100k training samples)
    \item Square root (6 digits, 100k training samples)
\end{itemize}

as well as tasks where the final model showed poor performance or did not even manage to go above zero accuracy:

\begin{itemize}
    \item Multiplication (3 digits, 100k training samples)
    \item Multiplication (5 digits, 100k training samples)
    \item Square root (20 digits, 100k training samples)
\end{itemize}

For each of these tasks, a separate error analysis dataset of 100k samples was created and the responses of the best model that was saved during training evaluated on it. A selection of the respective errors per task is presented below.

\subsubsection{Addition}

\includeTable
{
    task;model;correct\\
    91167+53933;145000;145100\\
    20518+99486;110004;120004\\
    68540+11493;070033;080033\\
    96725+23235;129960;119960\\
    58653+91352;140005;150005\\
    88018+94181;182299;182199\\
    73472+66524;149996;139996\\
    56590+43405;109995;099995\\
    68517+85882;154499;154399\\
    72822+97194;160016;170016\\
}
{tbl:errors_add_5digits}
{
    Model errors (addition task, 5 digits, 100k training dataset size).
}
{
    columns/task/.style={column name={Task}},
    columns/model/.style={column name={Model result}},
    columns/correct/.style={column name={Correct result}}
}

For addition with 5 operand digits, the model achieved near-perfect accuracy (\cref{tbl:baseline_add}).
It can be seen that the errors the model makes for this task (\cref{tbl:errors_add_5digits}) are single digits of the result being off by one.



\includeTable
{
    task;model;correct\\
    1470939445+3694860228;05165899673;05165799673\\
    9438411468+9410288290;18848799758;18848699758\\
    1600337712+5064162365;06664400077;06664500077\\
    8115628227+5759761714;13875399941;13875389941\\
    2578662692+8570333548;11149996240;11148996240\\
    8757570742+5604529187;14362199929;14362099929\\
    4847761430+3799620568;08647382998;08647381998\\
    7560389844+5150090143;12710489987;12710479987\\
    4277581356+7505693643;11783275999;11783274999\\
    7691931213+8322868827;16014700040;16014800040\\
}
{tbl:errors_add_10digits}
{
    Model errors (addition task, 10 digits, 100k training dataset size).
}
{
    columns/task/.style={column name={Task}},
    columns/model/.style={column name={Model result}},
    columns/correct/.style={column name={Correct result}}
}

A similar pattern emerges for 10-digit addition, which also achieved near-perfect accuracy. The failed tasks in \cref{tbl:errors_add_10digits} show a single digit, typically in the middle, being off by one. 

\subsubsection{Multiplication}

\includeTable
{
    task;model;correct\\
    896*551;493096;493696\\
    219*107;023933;023433\\
    192*926;177992;177792\\
    696*135;093060;093960\\
    787*809;636083;636683\\
    884*463;409392;409292\\
    131*907;118617;118817\\
    116*454;052464;052664\\
    763*738;563194;563094\\
    595*730;435350;434350\\
}
{tbl:errors_mul_3digits}
{
    Model errors (multiplication task, 3 digits, 100k training dataset size).
}
{
    columns/task/.style={column name={Task}},
    columns/model/.style={column name={Model result}},
    columns/correct/.style={column name={Correct result}}
}

For the multiplication task with 3 digits, the model achieved an accuracy of about 20\% (\cref{tbl:baseline_mul}). The errors are affecting only single digits, however some of the digits are off by more than one (\cref{tbl:errors_mul_3digits}).

\includeTable
{
    task;model;correct\\
    67767*25315;1717767705;1715521605\\
    10981*21880;0231011280;0240264280\\
    86313*14577;1251111101;1258184601\\
    98023*21001;2056666623;2058581023\\
    22900*56482;1295185800;1293437800\\
    41039*12284;0501111176;0504123076\\
    14038*12076;0164444688;0169522888\\
    55362*94976;5251717112;5258061312\\
    91694*73471;6731711774;6736849874\\
    90716*70758;6417716608;6418882728\\
}
{tbl:errors_mul_5digits}
{
    Model errors (multiplication task, 5 digits, 100k training dataset size).
}
{
    columns/task/.style={column name={Task}},
    columns/model/.style={column name={Model result}},
    columns/correct/.style={column name={Correct result}}
}

For the multiplication task with 5 digits, which did not achieve above-zero accuracy (\cref{tbl:baseline_mul}), the first and last digits were still correct for failed tasks, but multiple digits in the middle of the product were off (\cref{tbl:errors_mul_5digits}).

\subsubsection{Square root}

\includeTable
{
    task;model;correct;sqrt\\
    178980;422;423;423.06028\\
    458306;677;676;676.98301\\
    213411;462;461;461.96428\\
    139840;374;373;373.95187\\
    101789;318;319;319.04388\\
    128218;357;358;358.07541\\
    911019;953;954;954.47315\\
    451595;671;672;672.00818\\
    777861;882;881;881.96428\\
    254001;504;503;503.98512\\
}
{tbl:errors_sqrt_6digits}
{
    Model errors (square root task, 6 digits, 100k training dataset size).
}
{
    columns/task/.style={column name={Task}},
    columns/model/.style={column name={Model result}},
    columns/correct/.style={column name={Correct result}},
    columns/sqrt/.style={column name={$\sqrt{Task}$}}
}

\includeTable
{
    task;model;correct;sqrt\\
    79694857207412785081;8927155525;8927197612;8927197612.20803\\
    10138193304849869414;3183999089;3184052968;3184052968.28584\\
    20740203138915481021;4554099999;4554141317;4554141317.40721\\
    22785206268196001170;4773433272;4773385200;4773385200.06462\\
    61032565798178795501;7812323248;7812334209;7812334209.32430\\
    10321490749937914057;3212499999;3212707697;3212707697.55636\\
    24550288483602635175;4954848488;4954824768;4954824768.20348\\
    95746296144872950099;9785048434;9785003635;9785003635.40418\\
    76700026808202922573;8757999083;8757855148;8757855148.84797\\
    30130695889482657119;5489032224;5489143456;5489143456.81388\\
}
{tbl:errors_sqrt_20digits}
{
    Model errors (square root task, 20 digits, 100k training dataset size).
}
{
    columns/task/.style={column name={Task}},
    columns/model/.style={column name={Model result}},
    columns/correct/.style={column name={Correct result}},
    columns/sqrt/.style={column name={$\sqrt{Task}$}},
}


For the integer square root task with 6 operand digits, the model achieved about 99\% accuracy (\cref{tbl:baseline_sqrt}). The observed errors show the model answer being off by one from the correct answer (\cref{tbl:errors_sqrt_20digits}). Also, the real square root of the failed tasks was very close to an integer for 9 out of 10 failed tasks, which means the task was close to a square number, which are points from which on the integer square root increases by one. It is reasonable to assume that for such a "close call" it is harder to find the exact integer square root.

For the integer square root task with 20 operand digits, the model did not achieve above-zero accuracy. Analyzing some of the failed tasks shows the model predictions only have the leftmost digits correct, with about 4 out of 10 digits of the result being correct. Also, these tasks are not particularly close to a square number, which stands in contrast to the failed tasks for 6 operand digits.

\subsubsection{Conclusion}

We can see that the errors our model makes follow clear patterns: If the model is trained successfully (achieves good accuracy) the wrong answer is similar to the correct answer, with either a single digit being wrong in the case of addition and multiplication, or the square root being off by one in the case of integer square root. If the model is trained unsuccessfully (zero accuracy achieved) errors often affect more digits (as in the case of 5-digit multiplication).







\subsection{Model size experiments}
\label{model_size_experiments}



\subsubsection{Model size experiments: Addition}

\includeAccuracyTable
{experiment_results/model_size_final_accuracy/model_size.csv}
{tbl:model_size}
{Final model accuracy on the test dataset for the given operation and change in model configuration.}
{%
    columns/name/.append style={
        column name={},
        string replace={baseline}{baseline},
        string replace={n_embd=192}{$n_{embd}=192$},
        string replace={n_embd=768}{$n_{embd}=768$},
        string replace={n_head=3}{$n_{head}=3$},
        string replace={n_head=12}{$n_{head}=12$},
        string replace={n_layer=3}{$n_{layer}=3$},
        string replace={n_layer=12}{$n_{layer}=12$},
    },
    columns/add_10digits_100k/.style={column name={\makecell[r]{add\\(10 digits, 100k)}}},
    columns/mul_5digits_100k/.style={column name={\makecell[r]{mul\\(5 digits, 100k)}}},
}



For the addition task, we can see in \cref{tbl:model_size} that all of the tried model configurations were able to learn the task successfully, although there are small differences in final accuracy, such as the $n_{embd}=768$ model finishing with perfect accuracy on the test dataset.

When looking at the development of accuracy in \cref{fig:model_size_add}, we can see that there are substantial differences in how quickly the model converges on the addition task when changing the hidden vector size $n_{embd}$. For $n_{embd}=192$, it took about 2500 optimization steps for the model to cross 50\% accuracy, for $n_{embd}=384$ (baseline) it took about 900 steps and for $n_{embd}=768$ only about 500 steps. So in this case the bigger model with more parameters is able to learn faster than smaller models with fewer parameters.

Increasing or decreasing the number of GPT2 blocks/layers however only made a minor change in convergence speed, with 12 layers performing actually a bit worse than the baseline 6 layers.

Changing the number of attention heads, which does not impact the number of model parameters but only the "partitioning" of data inside the model, also had only a minor impact on convergence speed.

\includePDFPlot{experiment_results/model_size_accuracy/add.pdf}{fig:model_size_add}{Development of accuracy on the validation dataset when training for the addition task (10 digits, 100k dataset) for different changes in model configuration.}

\includePDFPlot{experiment_results/model_size_accuracy/mul.pdf}{fig:model_size_mul}{Development of accuracy on the validation dataset when training for the multiplication task (5 digits, 100k dataset) for different changes in model configuration.}


\subsubsection{Model size experiments: Multiplication}

For the multiplication task with 5 operand digits, we can see in \cref{tbl:model_size} and \cref{fig:model_size_mul} that none of the model size changes allowed the model to learn this harder task.







\subsection{Sampling methods}
\label{sampling methods}

\subsubsection{Results: Addition}

\begin{table}[H]
    \begin{minipage}[t]{0.47\linewidth}
        \hfill
        \includeAccuracyTableCore{experiment_results/sampling_strategies/basic_add.csv}{}
        \captionof{table}{Basic sampling.}
    \end{minipage}
    \hfill
    \begin{minipage}[t]{0.47\linewidth}
        \hfill
        \includeAccuracyTableCore{experiment_results/sampling_strategies/from_zero_add.csv}{}
        \captionof{table}{From-zero sampling.}
    \end{minipage}

    \vspace{0.5cm}
    \begin{minipage}[t]{0.47\linewidth}
        \hfill
        \includeAccuracyTableCore{experiment_results/sampling_strategies/uniform_digits_add.csv}{}
        \captionof{table}{Uniform digits sampling.}
    \end{minipage}
    \hfill
    \begin{minipage}[t]{0.47\linewidth}
        \hfill
        \includeAccuracyTableCore{experiment_results/sampling_strategies/uniform_bits_add.csv}{}
        \captionof{table}{Uniform bits sampling.}
    \end{minipage}
    \label{tbl:sampling_strategies_add}
\end{table}


For the addition task there were no dramatic differences in achieved model capability for the studied sampling methods  (\cref{tbl:sampling_strategies_add}). With a 1k training dataset size, none of the methods achieved high accuracy for any operand size, for 10k it was a mix with high accuracy for 3 operand digits and lower accuracy for 10 operand  digits. 100k dataset size allowed for good accuracy for all operand sizes.

There were, however, some less significant but still noticeable differences when comparing to the other sampling methods: Uniform digits and uniform bits sampling performed very similar to one another. Both achieved about 40\% accuracy for 3 digits, 1k dataset size while basic sampling and from-zero sampling remained close to zero. Here, the mix of easier (smaller, often single-digit) and harder (larger) training samples might have allowed for quicker convergence as we intended. On the other hand, they both had somewhat lower accuracy compared to basic sampling for 10k dataset size. From-zero sampling performed very similar to basic sampling,  except for 10 digits, where 10k training samples were not enough to achieve convergence. In that case, it could be that uniform digits and uniform bits sampling did not allow for enough training samples with the full 10 decimal digits in the training dataset if the training dataset size is small and the number of possible bits/digits is large.


\subsubsection{Results: Multiplication}

\begin{table}[H]
    \begin{minipage}[t]{0.47\linewidth}
        \hfill
        \includeAccuracyTableCore{experiment_results/sampling_strategies/basic_mul.csv}{}
        \captionof{table}{Basic sampling.}
    \end{minipage}
    \hfill
    \begin{minipage}[t]{0.47\linewidth}
        \hfill
        \includeAccuracyTableCore{experiment_results/sampling_strategies/from_zero_mul.csv}{}
        \captionof{table}{From-zero sampling.}
    \end{minipage}

    \vspace{0.5cm}
    \begin{minipage}[t]{0.47\linewidth}
        \hfill
        \includeAccuracyTableCore{experiment_results/sampling_strategies/uniform_digits_mul.csv}{}
        \captionof{table}{Uniform digits sampling.}
    \end{minipage}
    \hfill
    \begin{minipage}[t]{0.47\linewidth}
        \hfill
        \includeAccuracyTableCore{experiment_results/sampling_strategies/uniform_bits_mul.csv}{}
        \captionof{table}{Uniform bits sampling.}
    \end{minipage}
    \label{tbl:sampling_strategies_mul}
\end{table}

As expected, the multiplication task is much harder to learn, with most combinations of operand size and dataset size leaving the final model at zero accuracy. This does not change significantly based on the sampling method used, however uniform digits and uniform bits sampling do lead to a  slight increase in accuracy for the 100k-dataset, 3-digits case, which given the limited sample space and large number of samples, is the easiest case as expected.

\subsubsection{Results: Square root}


\begin{table}[H]
    \begin{minipage}[t]{0.47\linewidth}
        \hfill
        \includeAccuracyTableCore{experiment_results/sampling_strategies/basic_sqrt.csv}{}
        \captionof{table}{Basic sampling.}
    \end{minipage}
    \hfill
    \begin{minipage}[t]{0.47\linewidth}
        \hfill
        \includeAccuracyTableCore{experiment_results/sampling_strategies/from_zero_sqrt.csv}{}
        \captionof{table}{From-zero sampling.}
    \end{minipage}

    \vspace{0.5cm}
    \begin{minipage}[t]{0.47\linewidth}
        \hfill
        \includeAccuracyTableCore{experiment_results/sampling_strategies/uniform_digits_sqrt.csv}{}
        \captionof{table}{Uniform digits sampling.}
    \end{minipage}
    \hfill
    \begin{minipage}[t]{0.47\linewidth}
        \hfill
        \includeAccuracyTableCore{experiment_results/sampling_strategies/uniform_bits_sqrt.csv}{}
        \captionof{table}{Uniform bits sampling.}
    \end{minipage}
    \label{tbl:sampling_strategies_sqrt}
\end{table}

The integer square root task is somewhat in between addition and multiplication  for learning difficulty. The accuracy tables for basic and from-zero sampling also look very similar for this task.  Uniform digits and uniform bits sampling lead to slightly worse accuracy numbers, but the general characteristic is the same: The accuracy numbers mostly depend on the operation learned (addition vs multiplication vs square root) and only slightly change based on the sampling method.





\subsection{Step-by-step computation}
\label{stepbystep}



\subsubsection{Result reversal: Results}

\begin{table}[H]
	\begin{minipage}[t]{0.47\linewidth}
		\hfill
		\includeAccuracyTableCore{experiment_results/intermediate_steps/reverse/add.csv}{}
		\captionof{table}{Final model accuracy on the test dataset for the addition task (with result reversal) for the given training dataset sizes and operand digits.}
		\label{tbl:reverse_add}
	\end{minipage}
	\hfill %%%
	\begin{minipage}[t]{0.47\linewidth}
		\hfill
	    \includeAccuracyTableCore{experiment_results/baseline_final_accuracy/add.csv}{}
		\captionof{table}{Baseline additon accuracy (without result reversal).}
		\label{tbl:reverse_add_baseline}
	\end{minipage}
\end{table}

For the addition task (\cref{tbl:reverse_add}), when compared to the baseline experiments (\cref{tbl:reverse_add_baseline}), result reversal led to somewhat successful training (71\% accurracy) for the 3 digits, 1k dataset scenario, which could not be learned in baseline. However, training failed for the 10 digits, 10k dataset scenario.

To us it is not clear whether this is due to a true difference in the learnability between the two training sample structures, or simply due to variance between training runs. Similar differences observed in the sampling strategy experiments (\cref{tbl:sampling_strategies_add}) might point towards the latter.

It is also worth noting that for result reversal, 5 of the 9 scenarios, including all operand sizes for the 100k dataset size, led to perfect test accuracy (all 1000 samples correct), while for the baseline addition experiments accuracy was not quite perfect for the experiments with 5 and 10 operand digits.

\begin{table}[H]
	\begin{minipage}[t]{0.47\linewidth}
		\hfill
		\includeAccuracyTableCore{experiment_results/intermediate_steps/reverse/mul.csv}{}
		\captionof{table}{Final model accuracy on the test dataset for the multiplication task (with result reversal) for the given training dataset sizes and operand digits.}
		\label{tbl:reverse_mul}
	\end{minipage}
	\hfill %%%
	\begin{minipage}[t]{0.47\linewidth}
		\hfill
		\includeAccuracyTableCore{experiment_results/baseline_final_accuracy/mul.csv}{}
		\captionof{table}{Baseline multiplication accuracy (without result reversal).}
		\label{tbl:reverse_mul_baseline}
	\end{minipage}
\end{table}

For the multiplication task (\cref{tbl:reverse_mul}), results were in general very similar to those obtained without result reversal (\cref{tbl:reverse_mul_baseline}). This operation thus remained the hardest of the three studied and could not be made easier for the model by first presenting the digit-reversed product in the training samples.

\begin{table}[H]
	\begin{minipage}[t]{0.47\linewidth}
		\hfill
		\includeAccuracyTableCore{experiment_results/intermediate_steps/reverse/sqrt.csv}{}
		\captionof{table}{Final model accuracy on the test dataset for the square root task (with result reversal) for the given training dataset sizes and operand digits.}
		\label{tbl:reverse_sqrt}
	\end{minipage}
	\hfill %%%
	\begin{minipage}[t]{0.47\linewidth}
		\hfill
		\includeAccuracyTableCore{experiment_results/baseline_final_accuracy/sqrt.csv}{}
		\captionof{table}{Baseline square root accuracy (without result reversal).}
		\label{tbl:reverse_sqrt_baseline}
	\end{minipage}
\end{table}

For the square root task (\cref{tbl:reverse_sqrt}), the results were also very similar to the baseline task without result reversal (\cref{tbl:reverse_sqrt_baseline}), without significant differences in which scenarios could be learned and  the final accuracies obtained.



\subsubsection{Result reversal: Error analysis}

We wanted to investigate whether there was any difference in the error patterns between the baseline models and the ones trained using result reversal. To do this, we used the same tasks as for the baseline error analysis (\cref{error_analysis}): addition with 5 and 10 digits (100k dataset), multiplication with 3 and 5 digits (100k dataset) and square root with 6 and 20 digits (100k dataset).


\subsubsection{Result reversal: Error analysis (addition)}
\label{resutl_reversal_error_add}

For both the addition task with 5 digit and the addition task with 10 digit operands, when training using result reversal on a 100k training dataset, there were \textbf{no errors} in the entire 100k error analysis dataset, meaning the model performed each addition correctly. This corresponds to the perfect scores seen on the test dataset for these tasks (\cref{tbl:reverse_add}).
It stands in contrast to the baseline experiments, where despite 99\% accuracy, it was not hard to still find numbers where the addition was performed incorrectly (\cref{tbl:baseline_add}).

\subsubsection{Result reversal: Error analysis (multiplication)}

\includeTable
{
task;model;correct\\
896*551=;[692394]493296;[696394]493696\\
219*107=;[336320]023633;[334320]023433\\
192*926=;[293771]177392;[297771]177792\\
696*135=;[060390]093060;[069390]093960\\
787*809=;[382636]636283;[386636]636683\\
}
{tbl:reverse_errors_mul_3digits}
{
    Model errors (multiplication task, result reversal, 3 digits, 100k training dataset size).
}
{
    columns/task/.style={column name={Task}},
    columns/model/.style={column name={Model result}},
    columns/correct/.style={column name={Correct result}}
}

\includeTable
{
task;model;correct\\
67767*25315=;[5062665171]1715662605;[5061255171]1715521605\\
10981*21880=;[0826666320]0236666280;[0824620420]0240264280\\
86313*14577=;[1059999421]1249999501;[1064818521]1258184601\\
98023*21001=;[3211119502]2059111123;[3201858502]2058581023\\
22900*56482=;[0087774921]1294777800;[0087343921]1293437800\\
}
{tbl:reverse_errors_mul_5digits}
{
    Model errors (multiplication task, result reversal, 5 digits, 100k training dataset size).
}
{
    columns/task/.style={column name={Task}},
    columns/model/.style={column name={Model result}},
    columns/correct/.style={column name={Correct result}}
}

For the multiplication task with 3 digits (\cref{tbl:reverse_errors_mul_3digits}), the error patterns observed were basically the same as for the baseline experiments  (\cref{tbl:errors_mul_3digits}): Single digits being off by one or, sometimes, more than one.
The mistakes are already present in the reversed result,
which is then reversed correctly.

For the multiplication task with 5 digits (\cref{tbl:reverse_errors_mul_5digits}), we can also observe similar error patterns compared to the corresponding baseline experiment: The first and last digits of the result are correct, while the digits in the middle are wrong.
There also seems to be a pattern of repetition of single digits in the middle of the results, where the model has not learned to output the correct digits.

\subsubsection{Result reversal: Error analysis (square root)}

\includeTable
{
task;model;correct;sqrt\\
458306:;[776]677;[676]676;676.98301\\
478802:;[296]692;[196]691;691.95520\\
988048:;[399]993;[499]994;994.00604\\
651341:;[608]806;[708]807;807.05700\\
942826:;[179]971;[079]970;970.99228\\
}
{tbl:reverse_errors_sqrt_6digits}
{
    Model errors (square root task, result reversal, 6 digits, 100k training dataset size).
}
{
    columns/task/.style={column name={Task}},
    columns/model/.style={column name={Model result}},
    columns/correct/.style={column name={Correct result}},
    columns/sqrt/.style={column name={$\sqrt{Task}$}}
}

\includeTable
{
task;model;correct;sqrt\\
79694857207412785081:;[8828828298]8928288288;[2167917298]8927197612;8927197612.208\\
10138193304849869414:;[8888888813]3188888888;[8692504813]3184052968;3184052968.286\\
20740203138915481021:;[8828828554]4558288288;[7131414554]4554141317;4554141317.407\\
22785206268196001170:;[8288282774]4772828828;[0025833774]4773385200;4773385200.065\\
61032565798178795501:;[8818888187]7818888188;[9024332187]7812334209;7812334209.324\\
}
{tbl:reverse_errors_sqrt_20digits}
{
    Model errors (square root task, result reversal, 20 digits, 100k training dataset size).
}
{
    columns/task/.style={column name={Task}},
    columns/model/.style={column name={Model result}},
    columns/correct/.style={column name={Correct result}},
    columns/sqrt/.style={column name={$\sqrt{Task}$}}
}

For the 6-digit square root task (\cref{tbl:reverse_errors_sqrt_6digits}), the error patterns are also similar to the baseline (\cref{tbl:errors_sqrt_6digits}): Off-by-one errors on numbers where the operand is close to a square number, and the model "picked the wrong side".

For the 20-digit square root task (\cref{tbl:reverse_errors_sqrt_20digits}), the error patterns once again more or less match the baseline (\cref{tbl:errors_sqrt_20digits}): the 3 most significant digits are correct, the rest are wrong.
Compared to the baseline, we can see the result reversal model only got 3 out of 10 most significant digits correct for the examples in the table, whereas for the baseline it was 4 or even 5 correct digits for the examples.
Also there is an interesting digit pattern not seen in the baseline errors. A lot of repeated "8"s in the less significant digits where the model is wrong. This corresponds to similar digit repetitions for the multiplication task errors (\cref{tbl:reverse_errors_mul_5digits}).

\subsection{Step-by-step computation: Scratchpads}
\label{scratchpad}





\subsubsection{Scratchpads: Results}

\begin{table}[H]
	\begin{minipage}[t]{0.47\linewidth}
		\hfill
		\includeAccuracyTableCore{experiment_results/intermediate_steps/scratchpad/add.csv}{}
		\captionof{table}{Final model accuracy on the test dataset for the addition task (with scratchpad) for the given training dataset sizes and operand digits.}
		\label{tbl:scratchpad_add}
	\end{minipage}
	\hfill %%%
	\begin{minipage}[t]{0.47\linewidth}
		\hfill
		\includeAccuracyTableCore{experiment_results/baseline_final_accuracy/add.csv}{}
		\captionof{table}{Baseline addition accuracy (without scratchpad).}
		\label{tbl:scratchpad_add_baseline}
	\end{minipage}
\end{table}

For the addition task  (\cref{tbl:scratchpad_add}) there are clear improvements over the baseline (\cref{tbl:scratchpad_add_baseline}): For the 1k dataset size, training was more or less successful for all operand sizes (good accuracy but not 100\%), whereas there was no accuracy for the baseline experiments at 1k dataset size.
For 10k and 100k dataset sizes, there was perfect accuracy for all operand sizes, whereas in the baseline experiments accuracy was perfect only for 3 digit addition.

\begin{table}[H]
	\begin{minipage}[t]{0.47\linewidth}
		\hfill
		\includeAccuracyTableCore{experiment_results/intermediate_steps/scratchpad/mul.csv}{}
		\captionof{table}{Final model accuracy on the test dataset for the multiplication task (with scratchpad) for the given training dataset sizes and operand digits.}
		\label{tbl:scratchpad_mul}
	\end{minipage}
	\hfill %%%
	\begin{minipage}[t]{0.47\linewidth}
		\hfill
		\includeAccuracyTableCore{experiment_results/baseline_final_accuracy/mul.csv}{}
		\captionof{table}{Baseline multiplication accuracy (without scratchpad).}
		\label{tbl:scratchpad_mul_baseline}
	\end{minipage}
\end{table}


For the multiplication task (\cref{tbl:scratchpad_mul}), there were even more significant improvements made by providing the scratchpad: For the baseline experiments (\cref{tbl:scratchpad_mul_baseline}) only 3 digits, 10k dataset and 3 digits, 100k dataset provided above-zero accuracy, and even there it was only about 20\%. When using the scratchpad, accuracy is good for 
3 digits, 10k and 100k dataset sizes. Even for 5 and 10 digit multiplication, there is some degree of accuracy obtained, whereas it was a flat zero in the baseline experiments.

Because the accuracy numbers for 5 and 10 digits are somewhat "in the middle", showing the model is able to do a significant portion of multiplications correctly, but still making some errors, it is not unreasonable to assume training for longer would have made the model able to perform 5 and 10 digit multiplications with high accuracy. In contrast to the baseline, we had to limit training epochs for the multiplication scratchpad experiments to keep training times reasonable (\cref{tbl:scratchpad_training_mul}).

It is also an interesting observation that accuracies obtained on the 100k training dataset were lower than those on the 10k dataset. Training was limited to 50 epochs for the 10k dataset and 5 epochs for the 100k dataset (\cref{tbl:scratchpad_training_mul}). However, the total amount of training batches / optimization steps is the same for both scenarios, but the variety of training samples is lower in the 10k dataset scenario, increasing the risk of overfitting.
We would thus have expected better scores for the 100k dataset despite the lower epoch count.

\begin{table}[H]
	\begin{minipage}[t]{0.47\linewidth}
		\hfill
		\includeAccuracyTableCore{experiment_results/intermediate_steps/scratchpad/sqrt.csv}{}
		\captionof{table}{Final model accuracy on the test dataset for the square root task (with scratchpad) for the given training dataset sizes and operand digits.}
		\label{tbl:scratchpad_sqrt}
	\end{minipage}
	\hfill %%%
	\begin{minipage}[t]{0.47\linewidth}
		\hfill
		\includeAccuracyTableCore{experiment_results/baseline_final_accuracy/sqrt.csv}{}
		\captionof{table}{Baseline square root accuracy (without scratchpad).}
		\label{tbl:scratchpad_sqrt_baseline}
	\end{minipage}
\end{table}

For the square root task (\cref{tbl:scratchpad_sqrt}) the results were, interestingly, generally worse than baseline (\cref{tbl:scratchpad_sqrt_baseline}). This might be for a number of reasons:
First of all, we had to reduce the amount of training epochs, even more so than for the multiplication tasks, due to the very long scratchpads (\cref{tbl:scratchpad_sample_lengths}). Thus the resulting models are most likely severely undertrained.

Also, there is an important difference in the scratchpad structure between addition/multiplication and square root: For the addition and multiplication operations, the task was decomposed down into the smallest of pieces (single-digit addition and multiplication) (\cref{add_scratchpad}, \cref{mul_scratchpad}). Whereas for the square root task, we modeled the steps of the binary search algorithm but just left the squaring operations of large numbers as-is, without any explanation on how to compute the squares (we could not include this information as it would have made the scratchpads too massive) (\cref{sqrt_scratchpad}).

\subsubsection{Scratchpads: Error analysis}

Similar to the baseline and result reversal experiments, we looked for errors in the model output for the addition task with 5 and 10 digits (100k dataset), multiplication task with 3 and 5 digits (100k dataset) and square root task with 6 digits (100k dataset).
We did not analyze errors for the square root task with 20 operand digits, because the scratchpads are so long that presenting them here is impractical.
We only looked at 3 errors per task due to the longer output size when using the scratchpad. The error examples presented are simply the first 3 samples from the error analysis dataset where the model output did not match the correct output (considering both the scratchpad and final result).

\subsubsection{Scratchpads: Error analysis (addition)}

For the addition tasks with 5 and 10 digits and 100k training dataset, there were no errors in the 100k error analysis dataset, similar to training for addition using result reversal (\cref{resutl_reversal_error_add}).
Thus both result reversal and scratchpads allow for training a model that has a very low error probability for addition tasks up to 10 digits.

\subsubsection{Scratchpads: Error analysis (multiplication)}


\ErrorAnalysisTableBegin

\begin{lstlisting}
185*200=[
  *200{52001,82171,12130}37000,
  *0{50000,80000,10000}0,
  *0{50000,80000,10000}0
|{
  000,0,0,0;
  0,0,0,0;
  0,0,0,0;
  !3!,0,!3!,0;
  3,0,3,0
}]
3!3!000
\end{lstlisting} &
\begin{lstlisting}
185*200=[
  *200{52001,82171,12130}37000,
  *0{50000,80000,10000}0,
  *0{50000,80000,10000}0
|{
  000,0,0,0;
  0,0,0,0;
  0,0,0,0;
  7,0,7,0;
  3,0,3,0
}]
37000
\end{lstlisting} \\ \ErrorAnalysisTableRule

\begin{lstlisting}
595*730=[
  *700{57053,97366,57614}416500,
  *30{53051,93182,53271}17850,
  *0{50000,90000,50000}0
|{
  000,0,0,0;
  05,0,5,0;
  58,0,3,1;
  67,1,4,1;
  11,1,3,0;
  4,0,4,0
}]
4!5!4350
\end{lstlisting} &
\begin{lstlisting}
595*730=[
  *700{57053,97366,57614}416500,
  *30{53051,93182,53271}17850,
  *0{50000,90000,50000}0
|{
  000,0,0,0;
  05,0,5,0;
  58,0,3,1;
  67,1,4,1;
  11,1,3,0;
  4,0,4,0
}]
434350
\end{lstlisting} \\ \ErrorAnalysisTableRule

\begin{lstlisting}
577*170=[
  *100{71070,71070,51050}57700,
  *70{77094,77435,57504}40390,
  *0{70000,70000,50000}0
|{
  000,0,0,0;
  09,0,9,0;
  73,0,0,1;
  70,1,8,0;
  54,0,9,0
}]
980!8!0
\end{lstlisting} &
\begin{lstlisting}
577*170=[
  *100{71070,71070,51050}57700,
  *70{77094,77435,57504}40390,
  *0{70000,70000,50000}0
|{
  000,0,0,0;
  09,0,9,0;
  73,0,0,1;
  70,1,8,0;
  54,0,9,0
}]
98090
\end{lstlisting} \\

\ErrorAnalysisTableEnd{tbl:errors_scratchpad_mul3}{Error examples for the multiplication task (3 digits, 100k training dataset, using scratchpad, trained for 5 epochs).}

For the 3-digit multiplication task (\cref{tbl:errors_scratchpad_mul3}), we can see there were no errors in the single-digit multiplications. In the first error sample, the model fetched the wrong digit from the single-digit results during the addition phase. We hypothesize this could be due to there being multiple single-digit products that are zero, thus making it harder to pick the right digits due to the uneven length of the products.

For the other two error samples, the model did the entire calculation correctly, but surprisingly failed to read the correct digit sequence from the addition scratchpad.


\ErrorAnalysisTableBegin

\begin{lstlisting}
67767*25315=[
  *20000{72041,62131,72151,72151,62131}
    1355340000,
  *5000{75053,65333,75383,75383,65333}
    338835000,
  *300{73012,63202,73232,73232,63202}
    20330100,
  *10{71070,61060,71070,71070,61060}
    677670,
  *5{75053,65333,75383,75383,65333}
    338835
|{
  00005,0,5,0;
  00073,0,0,1;
  00168,1,6,1;
  05078,1,1,2;
  43373,2,!3!,2;
  38363,2,!7!,2;
  580,2,5,1;
  532,1,1,1;
  33,1,7,0;
  1,0,1,0
}]
1715!73!1605
\end{lstlisting} &
\begin{lstlisting}
67767*25315=[
  *20000{72041,62131,72151,72151,62131}
    1355340000,
  *5000{75053,65333,75383,75383,65333}
    338835000,
  *300{73012,63202,73232,73232,63202}
    20330100,
  *10{71070,61060,71070,71070,61060}
    677670,
  *5{75053,65333,75383,75383,65333}
    338835
|{
  00005,0,5,0;
  00073,0,0,1;
  00168,1,6,1;
  05078,1,1,2;
  43373,2,2,2;
  38363,2,5,2;
  580,2,5,1;
  532,1,1,1;
  33,1,7,0;
  1,0,1,0
}]
1715521605
\end{lstlisting} \\ \ErrorAnalysisTableRule
\ErrorAnalysisTableEnd{tbl:errors_scratchpad_mul5_1}{Error example 1 for the multiplication task (5 digits, 100k training dataset, using scratchpad, trained for 5 epochs).}

\ErrorAnalysisTableBegin
\begin{lstlisting}
10981*21880=[
  *20000{12020,82061,92191,02110,12020}
    219620000,
  *1000{11010,81080,91090,01000,11010}
    10981000,
  *800{18080,88046,98687,08770,18080}
    8784800,
  *80{18080,88046,98687,08770,18080}
    878480,
  *0{10000,80000,90000,00000,10000}
    0
|{
  00000,0,0,0;
  0008,0,8,0;
  0084,0,2,1;
  0148,1,4,1;
  2887,1,!8!,2;
  6978,2,!0!,3;
  908,3,0,2;
  11,2,4,0;
  2,0,2,0
}]
240!08!4280
\end{lstlisting} &
\begin{lstlisting}
10981*21880=[
  *20000{12020,82061,92191,02110,12020}
    219620000,
  *1000{11010,81080,91090,01000,11010}
    10981000,
  *800{18080,88046,98687,08770,18080}
    8784800,
  *80{18080,88046,98687,08770,18080}
    878480,
  *0{10000,80000,90000,00000,10000}
    0
|{
  00000,0,0,0;
  0008,0,8,0;
  0084,0,2,1;
  0148,1,4,1;
  2887,1,6,2;
  6978,2,2,3;
  908,3,0,2;
  11,2,4,0;
  2,0,2,0
}]
240264280
\end{lstlisting} \\
\ErrorAnalysisTableEnd{tbl:errors_scratchpad_mul5_2}{Error example 2 for the multiplication task(5 digits, 100k training dataset, using scratchpad, trained for 5 epochs).}


\ErrorAnalysisTableBegin
\begin{lstlisting}
98023*21001=[
  *20000{32060,22040,02000,82061,92191}
    1960460000,
  *1000{31030,21020,01000,81080,91090}
    98023000,
  *0{30000,20000,00000,80000,90000}
    0,
  *0{30000,20000,00000,80000,90000}
    0,
 *1{31030,21020,!2!10!2!0,!0!1000,!8!10!8!0}
    !802!23
|{
  00003,0,3,0;
  002,0,2,0;
  00!2!,0,!2!,0;
  03!0!,0,!3!,!0!;
  62!8!,0,!6!,!1!;
  40,1,5,0;
  08,0,8,0;
  69,0,5,1;
  9,1,0,1;
  1,1,2,0
}]
20585!632!23
\end{lstlisting} &
\begin{lstlisting}
98023*21001=[
  *20000{32060,22040,02000,82061,92191}
    1960460000,
  *1000{31030,21020,01000,81080,91090}
    98023000,
  *0{30000,20000,00000,80000,90000}
    0,
  *0{30000,20000,00000,80000,90000}
    0,
 *1{31030,21020,01000,81080,91090}
    98023
|{
  00003,0,3,0;
  002,0,2,0;
  000,0,0,0;
  038,0,1,1;
  629,1,8,1;
  40,1,5,0;
  08,0,8,0;
  69,0,5,1;
  9,1,0,1;
  1,1,2,0
}]
2058581023
\end{lstlisting}
\\
\ErrorAnalysisTableEnd{tbl:errors_scratchpad_mul5_3}{Error example 3 for the multiplication task(5 digits, 100k training dataset, using scratchpad, trained for 5 epochs).}

For the multiplication task with 5 operand digits, the first example (\cref{tbl:errors_scratchpad_mul5_1}) shows the model failing to compute the correct sum based on the correct list of digits and current carry value. This led to some digits in the middle of the product being wrong.

The second example  (\cref{tbl:errors_scratchpad_mul5_2}) shows the same pattern: wrong sum adding multiple digits during the addition phase - leading to some wrong digits in the middle of the final product.

The third example  (\cref{tbl:errors_scratchpad_mul5_3}) shows a somewhat different pattern: Interestingly, there were mistakes in a single-digit multiplication with factor 1, basically failing to copy the digits. This led to wrong digits in the middle of the final product.

In summary, there was only one error observed in the digit multiplication stage and one error writing the final result, while there were multiple errors in the sum stage, either computing the wrong sum or fetching the wrong digits.


\subsubsection{Scratchpads: Error analysis (square root)}



\ErrorAnalysisTableBegin
\begin{lstlisting}
915965:[
{0,915965}457982*457982=209!199148!324
{0,457982}228991*228991=52!299787!081
{0,228991}114495*114495=13!074832!025
{0,114495}57247*57247=32!68650!009
{0,57247}28623*28623=81!7133!129
{0,28623}14311*14311=204!268!721
{0,14311}7155*7155=51!060!025
{0,7155}3577*3577=127!61!929
{0,3577}1788*1788=31!88!944
{0,1788}894*894=799236
{894,1788}1341*1341=1798281
{894,1341}1117*1117=1247689
{894,1117}1005*1005=1010025
{894,1005}949*949=900601
{949,1005}977*977=954529
{949,977}963*963=927369
{949,963}956*956=913936
{956,963}959*959=919681
{956,959}957*957=915849
{!956,957!}957*957=915849
{!956,957!}
]
!956!
\end{lstlisting} &
\begin{lstlisting}
915965:[
{0,915965}457982*457982=209747512324
{0,457982}228991*228991=52436878081
{0,228991}114495*114495=13109105025
{0,114495}57247*57247=3277219009
{0,57247}28623*28623=819276129
{0,28623}14311*14311=204804721
{0,14311}7155*7155=51194025
{0,7155}3577*3577=12794929
{0,3577}1788*1788=3196944
{0,1788}894*894=799236
{894,1788}1341*1341=1798281
{894,1341}1117*1117=1247689
{894,1117}1005*1005=1010025
{894,1005}949*949=900601
{949,1005}977*977=954529
{949,977}963*963=927369
{949,963}956*956=913936
{956,963}959*959=919681
{956,959}957*957=915849
{957,959}958*958=917764
{957,958}
]
957
\end{lstlisting} \\
\ErrorAnalysisTableEnd{tbl:errors_scratchpad_sqrt6_1}{Error example 1 for the square root task, (6 digits, 100k training dataset, using scratchpad, trained for 5 epochs).}


\ErrorAnalysisTableBegin
\begin{lstlisting}
195046:[
{0,195046}97523*97523=95!0818!5529
{0,97523}48761*48761=2377!097!121
{0,48761}24380*24380=594!24!4400
{0,24380}12190*12190=1485!616!00
{0,12190}6095*6095=3714!0!025
{0,6095}3047*3047=928!1!209
{0,3047}1523*1523=231!8!529
{0,1523}761*761=579121
{0,761}380*380=144400
{380,761}570*570=324900
{380,570}475*475=225625
{380,475}427*427=182329
{427,475}451*451=203401
{427,451}439*439=192721
{439,451}445*445=198025
{439,445}442*442=195364
{439,442}440*440=193600
{440,442}441*441=194481
{!440,441!}
]
!440!
\end{lstlisting} &
\begin{lstlisting}
195046:[
{0,195046}97523*97523=9510735529
{0,97523}48761*48761=2377635121
{0,48761}24380*24380=594384400
{0,24380}12190*12190=148596100
{0,12190}6095*6095=37149025
{0,6095}3047*3047=9284209
{0,3047}1523*1523=2319529
{0,1523}761*761=579121
{0,761}380*380=144400
{380,761}570*570=324900
{380,570}475*475=225625
{380,475}427*427=182329
{427,475}451*451=203401
{427,451}439*439=192721
{439,451}445*445=198025
{439,445}442*442=195364
{439,442}440*440=193600
{440,442}441*441=194481
{441,442}
]
441
\end{lstlisting} \\
\ErrorAnalysisTableEnd{tbl:errors_scratchpad_sqrt6_2}{Error example 2 for the square root task (6 digits, 100k training dataset, using scratchpad, trained for 5 epochs).}

\ErrorAnalysisTableBegin
\begin{lstlisting}
562141:[
{0,562141}281070*281070=7!817572!4900
{0,281070}140535*140535=19!543931!225
{0,140535}70267*70267=4!885917!289
{0,70267}35133*35133=12!21444!689
{0,35133}17566*17566=30!5343!356
{0,17566}8783*8783=7!6335!089
{0,8783}4391*4391=19!079!881
{0,4391}2195*2195=4!767!025
{0,2195}1097*1097=1!190!409
{0,1097}548*548=!299!304
{548,1097}822*822=675684
{548,822}685*685=469225
{685,822}753*753=567009
{685,753}719*719=516961
{719,753}736*736=541696
{736,753}744*744=553536
{744,753}748*748=559504
{748,753}750*750=562500
{748,750}749*749=561001
{749,750}
]
749
\end{lstlisting} &
\begin{lstlisting}
562141:[
{0,562141}281070*281070=79000344900
{0,281070}140535*140535=19750086225
{0,140535}70267*70267=4937451289
{0,70267}35133*35133=1234327689
{0,35133}17566*17566=308564356
{0,17566}8783*8783=77141089
{0,8783}4391*4391=19280881
{0,4391}2195*2195=4818025
{0,2195}1097*1097=1203409
{0,1097}548*548=300304
{548,1097}822*822=675684
{548,822}685*685=469225
{685,822}753*753=567009
{685,753}719*719=516961
{719,753}736*736=541696
{736,753}744*744=553536
{744,753}748*748=559504
{748,753}750*750=562500
{748,750}749*749=561001
{749,750}
]
749
\end{lstlisting} \\
\\
\ErrorAnalysisTableEnd{tbl:errors_scratchpad_sqrt6_3}{Error example 3 for the square root task (6 digits, 100k training dataset, using scratchpad, trained for 5 epochs).}

For the square root task with 6 digits, in the first example (\cref{tbl:errors_scratchpad_sqrt6_1}) we can see the model making mistakes in squaring large numbers at the beginning of the algorithm. This, however, did not influence the final result as the algorithm always has to take a few steps in the same direction at the start due do the large search interval $[0, n]$.
The model did make another mistake at the very end of the algorithm, picking the wrong half of the search space and then picking the wrong one of two numbers for the final square root.

In the second example (\cref{tbl:errors_scratchpad_sqrt6_2}) we can see the model making similar errors computing large squares at the start. It then did fine computing smaller squares and picking the right half of the search space. At the very end, it picked the wrong half of the search space, similarly to the first example.

Example 3 (\cref{tbl:errors_scratchpad_sqrt6_3}) shows the model coming to the right conclusion after making a few mistakes computing larger squares at the start.

In summary, we can see that errors happen when computing larger square numbers at the start of the algorithm and when picking the right half at the end of the algorithm. Interestingly, computing squares towards of the algorithm does not seem a cause of errors.


\subsection{Step-by-step computation: Accuracy development}

We wanted to investigate how quickly the model achieved good accuracy for the various tasks and operand sizes when using result reversal or scratchpad compared to baseline. For this, we plotted the accuracy on the validation dataset during training.

The diagrams only show accuracy development for the 100k dataset size for brevity.

For scratchpad-based training, we evaluated accuracy only once every epoch and reduced the validation dataset to 10 samples (\cref{scratchpad_changes}). This of course leads to less precise curves, but can still give a rough idea about training progress.

\includePDFPlot{experiment_results/intermediate_steps_accuracy/add.pdf}{fig:intermediate_steps_accuracy_add}{Development of accuracy on the validation dataset, when training the addition task using a 100k dataset using various training methods and operand digits.}

For the addition task (\cref{fig:intermediate_steps_accuracy_add}) we can see that baseline training needs between 2000 (3 digits) and over 20000 (10 digits) training steps to achieve near 100\% accuracy. In contrast, when using result reversal, the model learned somewhat faster (1000-2000 steps for 3 and 5 digits, 10000 steps for 10 digits) and accuracy hits 100\% much more abruptly compared to baseline, where it hits 80\% and then needs some time to get to 100\%.

Using a scratchpad leads to much faster convergence, where 3 digit addition already converged at the first log interval at 400 steps, and 5 and 10 digit addition converged before 1000 steps.




\includePDFPlot{experiment_results/intermediate_steps_accuracy/mul.pdf}{fig:intermediate_steps_accuracy_mul}{Development of accuracy on the validation dataset, when training the multiplication task using a 100k dataset using various training methods and operand digits.}

For the multiplication task (\cref{fig:intermediate_steps_accuracy_mul}), we can see that only scratchpad with 3 digit multiplication converged to 100\%. For the same task, baseline and result reversal had similar curves, starting to leave zero at about 3000 steps, and ending at around 20\% accuracy after about 40000 steps.

For 5 and 10 digit multiplication, only scratchpad-based training achieved above-zero accuracy, while still remaining at around 20\% accuracy and taking between 3000 and 20000 steps to start converging.
It might however converge all the way to 100\% given continued training, but resource constraints made us limit the number of training epochs (\cref{scratchpad_changes}).

\includePDFPlot{experiment_results/intermediate_steps_accuracy/sqrt.pdf}{fig:intermediate_steps_accuracy_sqrt}{Development of accuracy on the validation dataset, when training the square root task using a 100k dataset using various training methods and operand digits.}

For the square root task with 6 operand digits (\cref{fig:intermediate_steps_accuracy_sqrt}), baseline and result reversal performed very similar while the scratchpad took longer to start converging but then quickly converged, matching the other two training methods.

For 10 operand digits, scratchpad actually failed to converge, as opposed to baseline and result reversal. This happened despite taking more optimization steps until training was stopped. It would thus be reasonable to assume that the form of scratchpad we used here, which does not further elaborate on the squaring step, is not helpful for model training, as opposed to the successful addition and multiplication scratchpads.





\subsection{Attention heatmaps}
\label{heatmap}


\subsubsection{How to interpret the diagrams}

\begin{itemize}
	\item Each row refers to the sequence position where a certain token is generated by the model (the position in the token sequence where the model outputs this token based on the previous ones as input).
	
	\item For each row, the columns of that row refer to previous token positions that the model will consider for generating the output at that position. Note that the attention values for the current and following token positions are always zero, as the model has not yet generated these values at that point.
	
	\item Colors show attention intensity and range from dark purple (no attention to this position) to yellow (full attention to this position).
\end{itemize}

\subsubsection{Addition task}

\includePDFPlot{experiment_results/heatmaps/add_full_1.pdf}{fig:heatmap_add_full}{Attention activation heatmap for all blocks and attention heads for the addition task "123+456".}

Attention activation for the addition task "123+456=0579" is illustrated in \cref{fig:heatmap_add_full} for each of the 6 GPT2 blocks and each of the 6 attention heads per block.

When looking for patterns in these activation heatmaps, a clear relationship is visible in Block 1, where the position of the output digit "5" is mostly attending to the positions to the input digits "1" and "4", the position of the output digit "7" to the input digits "2" and "5" and the output digit "9" to the input digits "3" and "6". The first output digit "0" also attends mostly to the input digits "1" and "4". This is consistent with how data would flow from input to output in a standard addition algorithm. The stop token "." position does not attend to any input position in particular.
This pattern is more or less the same across all 6 attention heads.

When looking at the later blocks, there are also some spots of high activation, however these are often not uniform across attention heads and there is no obvious pattern between input and outputs.

\subsubsection{Single-heatmap diagrams}

Since we found out that the most visible attention patterns are in the first block, to keep the diagrams of reasonable size especially for larger sequences, for the rest of the diagrams only the attention mechanism of the first GPT2 block is considered, and the attention values of all 6 heads are averaged.

\includePDFPlot{experiment_results/heatmaps/add_simple_1.pdf}{fig:heatmap_add_simple1}{First block attention heatmap for the addition task "123+456", averaged over all 6 heads.}

The single-diagram heatmap of \cref{fig:heatmap_add_simple1} for the same addition task shows the same pattern mentioned before, with an output digit being connected to those two input digits being added for this position, with the relationship being clear for the digits "579" and a bit less prominent for the digit "0".

\includePDFPlot{experiment_results/heatmaps/add_simple_2.pdf}{fig:heatmap_add_simple2}{First block attention heatmap for the addition task "999+001", averaged over all 6 heads.}

There is little change to the previous diagram in \cref{fig:heatmap_add_simple2} which shows the same addition model working on different numbers, namely the addition "999+001". Only minor changes in attention value are visible when doing addition of different numbers using the same model.

\includePDFPlot{experiment_results/heatmaps/add_5digits_1.pdf}{fig:heatmap_add_5digits}{First block attention heatmap for the addition task "12345+67890", averaged over all 6 heads.}

When switching to a model trained for 5-digit addition and the addition task "12345+67890" (\cref{fig:heatmap_add_5digits}) we can still see a similar pattern of output digits referring to those input digits that make sense when thinking about an addition algorithm. However the pattern is somewhat more diffuse for the more significant output digits, which are also attending to less significant digits of the input numbers somewhat.

\includePDFPlot{experiment_results/heatmaps/add_rev_1.pdf}{fig:heatmap_add_rev}{First block attention heatmap for the addition task "123+456", using result reversal, averaged over all 6 heads.}

For 3-digit addition using result reversal (\cref{fig:heatmap_add_rev}), we can see a very clear pattern of the final output digits attending to the corresponding digits of the reversed result. There however is not such a clear pattern for the reversed addition that happens before, with the output digits "9750" not having a strong connection to the matching input digits as was seen before for normal addition.

\includePDFPlot{experiment_results/heatmaps/add_scratch_1.pdf}{fig:heatmap_add_scratch}{First block attention heatmap for the addition task "123+456", using result reversal, averaged over all 6 heads.}

For the addition task using the scratchpad (\cref{fig:heatmap_add_scratch}), we can also see some patterns: For the individual addition steps, the input digits are fetched as the first two digits of each step. We can also see that the old carry value (third digit of each step) to some degree attends to the fifth digit of the previous step, which is the output carry value of that previous step.

\subsubsection{Multiplication task}

\includePDFPlot{experiment_results/heatmaps/mul_1.pdf}{fig:heatmap_mul}{First block attention heatmap for the multiplication task "123*456", averaged over all 6 heads.}

For the multiplication task with 3 digits (\cref{fig:heatmap_mul}) our model actually computes the wrong result 056188 instead of the correct result 056088 which is not unexpected considering its low accuracy at about 20\% (\cref{tbl:baseline_mul}).

We can see the 3 least significant digits of the product (188) attending primarily to the  least significant digits of the input numbers (3 and 6). The 3rd most significant digit (6) has an attention pattern without clear focus. The 1st and 2nd most significant digits of the product (0 and 5) attend to the most significant digits of the input (1 and 4).

This does not follow a pattern clearly related to a normal multiplication algorithm: For the least significant digit of the product only the least significant digit of the factors should be relevant, for the second least significant digit this should extend to the first and second digit of the factors and so on, which does not match the pattern visible in the diagram.
This corresponds well to the fact that accuracy achieved for multiplication was much poorer that for addition (only 22\% for 3-digit multiplication, see \cref{tbl:baseline_mul}), prompting speculation whether models with high accuracy on an arithmetic task show more meaningful digit relationships in their heatmaps that those with low accuracy.

\subsubsection{Square root task}

\includePDFPlot{experiment_results/heatmaps/sqrt_1.pdf}{fig:heatmap_sqrt}{First block attention heatmap for the integer square root task "123456", averaged over all 6 heads.}

For the square root tasks with 6 digits (\cref{fig:heatmap_sqrt}) we can see the least significant digit of the output "1" attending to the 3 most significant digits of the input (and slightly to previous output digit "3") whereas the more significant output digit 5 attends mostly to the second most significant input digit "2" and the most significant output digit "3" attends mostly to the most significant input digit "1".

This also does not clearly correspond to a normal integer square root algorithm as for example the least significant 3 digits of the input are not considered at all by the model's attention pattern, while for example 123000 has an integer square root of 350 whereas 123999 has a square root of 352. These digits must thus be considered, if they are even considered at all, only at later blocks of the model not visible in this diagram.

Accuracy was 98\% for 6-digit integer square root (\cref{tbl:baseline_sqrt}), but for this operation it is plausible that the model just guesses the square root based on the first few digits, which would of course fail on numbers close to a square number which require precise computation, but such hard cases should be only a small fraction of the test dataset (which is uniformly selected from all 6-digit integers).


%\section{Summary}

As a baseline, we trained a small transformer network on addition, multiplication and integer square root tasks, with different numbers of operand digits and training dataset sizes (\cref{results:baseline}).

We wanted to improve performance, in particular for the multiplication and integer square root tasks and thus attempted a number of changes to the model and dataset.

First, we explored whether changing the model size (number of layers, hidden vector sizes etc.) improved model performance (\cref{results:modelsize}).
After that, we tried changing the sampling function for the training dataset to see whether prioritizing certain training samples improved performance (\cref{results:sampling}).
Then, we went on to change the actual training samples themselves. For this, we tried two different strategies:
The first strategy was reversing the digits of the result for each training sample (\cref{results:reversal}).
The second strategy was augmenting each training sample with a scratchpad of intermediate steps of the arithmetic computation of that sample (\cref{results:scratchpad}).
Finally, we created heatmaps of model attention to get some insight into how the model processes input strings (\cref{results:heatmap}).

\subsection{Results}
When doing the baseline experiments (\cref{results:baseline}), we found out that addition was generally easily learned as long as the training dataset size was at least 10k (\cref{fig:baseline_add}), with accuracy of at least 99\% for 3, 5 and 10 digit addition when using a 100k training dataset. These results are in line with \cite{teaching}.
Multiplication proved much tougher, with only about 20\% accuracy achieved for 3-digit multiplication and zero accuracy for 5 and 10 digits (\cref{fig:baseline_mul}).
Integer square root, for a comparable operand size (single operand with twice as many digits as the two operands of addition or multiplication) proved to be somewhat in the middle in terms of difficulty. A 100k training dataset achieved 98\% accuracy for 6 digit integer square root and 64\% accuracy for 10 digit integer square root, with zero accuracy for 20 digit integer square root (\cref{fig:baseline_sqrt}).

Errors for addition and multiplication were generally single/multiple digits being off in the middle of the sum/product (\cref{error_analysis}). This matches the error patterns observed in \citepage{7}{lengthgen}, which shows digits in the middle of the result taking longer to converge and finishing at lower accuracy.
For square root, the errors where the result being off by one for numbers close to a square number.

Changing the model size did result in a model with bigger vectors or more layers needing fewer training steps to achieve the same accuracy (\cref{fig:model_size_add}, \cref{fig:model_size_mul}) but did not make a substantial difference in model capabilities, meaning it could still not learn multiplication with 5 or 10 digits, or integer square root with 20 digits.

Changing the digit sampling from uniform across the whole sample space to bit-uniform or digit-uniform provided little improvement either. This stands in contrast to \cite{positionmatters} where both uniform-digit sampling and result reversal provided some improvement in model accuracy.

Result reversal did improve accuracy for addition when using a 100k training dataset: from 98-99\% to 100\% (\cref{tbl:reverse_add}). While it did increase reliability for addition, it did not help with making large multiplication or integer square root learnable.

Scratchpads led to big improvements for addition and multiplication: Addition could be learned perfectly with a 10k and 100k training dataset, and even for a 1k training dataset, accuracy numbers ranged between 85\% and 99\% (\cref{tbl:scratchpad_add}). For multiplication, accuracy numbers of 94\% for 3-digit multiplication, 63\% for 5-digit multiplication, and 22\% for 10-digit multiplication could be achieved. Interestingly, these numbers were obtained with a 10k training dataset, with slightly lower numbers achieved for the 100k training dataset. For integer square root, scratchpad performed worse than baseline (\cref{tbl:scratchpad_sqrt}) - this could be due to our square root scratchpad not containing each and every intermediate step due to size constraints.

When looking at the heatmaps for model attention, we could see some meaningful relationships between input and output digits in the first layer, especially for addition, corresponding to how addition would be done by hand (\cref{fig:heatmap_add_full}, \cref{fig:heatmap_add_simple1}). What happened in later layers was not obvious in the visualization. When using result reversal and scratchpad, there was a clear connection between input digits and these same digits in the intermediate steps of the computation, but no clear pattern of the rest of the computation (\cref{fig:heatmap_add_rev}, \cref{fig:heatmap_add_scratch}).

In summary, the best results we could achieve for the arithmetic functions we studied were 100\% accuracy for 3, 5 and 10 digit addition, 94\% for 3-digit multiplication, 63\% for 5-digit multiplication, 22\% for 10-digit multiplication, 98\% for 6-digit integer square root, 64\% for 10-digit integer square root and 0\% for 20-digit integer square root.
For addition and multiplication, these results were obtained when using scratchpads, for integer square root, they are from the baseline model.

On the one hand, these results could likely be made even better by longer training when using scratchpads (we constrained epochs to keep training time manageable with the long scratchpad strings).
On the other hand, using scratchpads also requires knowledge of meaningful intermediate steps - how to compute the output from the input step by step. A key advantage of machine learning over traditional algorithms is that this knowledge of how to compute the result is generally not required: The model is provided input/output pairs and figures out the rest. This advantage is lost when needing to build a detailed scratchpad for the training samples.



\subsection{Future work}

Adding to the results of our work could start by running training on the scratchpad experiments for longer, until an actual plateau is reached. Training using scratchpad was limited by resources available to us. Training for more epochs would allow to see the actual limitations of scratchpad training for multiplication on our model better.
It is also unclear whether the stopping condition we used does lead to premature termination of training - looking at accuracy curves suggests this might be a possibility. Focusing not only on loss but also accuracy might ensure training does not stop before the model reaches maximum accuracy.

We did not analyze variance between multiple training runs as even a single training run took substantial resources. It would however be interesting to see if a task that normally fails to converge occasionally does converge or vice versa.
It would be worth combining result reversal with uniform-digit sampling to try and replicate the good results for multiplication in \cite{positionmatters}.
It would also be interesting to see whether a second model could learn to predict where the main model makes a mistake and try to correct it, especially for longer scratchpads

The good results of basic feed-forward networks in \cite{visual} would make it worthwhile to explore their limits, especially with regards to longer scratchpad sequences. It would be worth finding out at which point transformers start having an advantage over older, more basic networks. Other structures like 1-dimensional CNNs or recurrent networks could be compared too.

Using a simpler, more straightforward scratchpad for multiplication as in \cite{implicit} would reduce sequence length and thus accelerate training. Performance between such a bare-bones scratchpad and the more elaborate one we are using should be compared.
At the same time, a better scratchpad for integer square root should be devised. It must be reasonably short, yet not leave big chunks without intermediate steps, like ours does for squaring integers.

Finally, the inner mechanisms of the model could be explored further. Our heatmaps show clear, logical relationships between input and output in the first layer already - but it does not explain the whole "algorithm" learned by the model. Analysis would have to focus on the more opaque patterns in later layers of the model.


\printbibliography[nottype=online]

\clearpage
\printbibliography[type=online, title={Links}]

\end{document}